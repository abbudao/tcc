% ------------------------------------------------------------------------
% ------------------------------------------------------------------------
% ICMC: Modelo de Trabalho Acadêmico (tese de doutorado, dissertação de
% mestrado e trabalhos monográficos em geral) em conformidade com 
% ABNT NBR 14724:2011: Informação e documentação - Trabalhos acadêmicos -
% Apresentação
% ------------------------------------------------------------------------
% ------------------------------------------------------------------------

% Opções: 
%   Qualificação         = qualificacao 
%   Curso                = doutorado/mestrado
%   Situação do trabalho = pre-defesa/pos-defesa (exceto para qualificação)
% -- opções do pacote babel --
% Idioma padrão = brazil
	%french,	    	% idioma adicional para hifenização
	%spanish,			% idioma adicional para hifenização
	%english,			% idioma adicional para hifenização
	%brazil				% o último idioma é o principal do documento
	\documentclass[brazil]{packages/icmc}

% ---
% Pacotes Opcionais
% ---
\usepackage{minted}           % Usado para rotacionar o texto
\usepackage{rotating}           % Usado para rotacionar o texto
\usepackage[all,knot,arc,import,poly]{xy}   % Pacote para desenhos gráficos
% Este pacote pode conflitar com outros pacotes gráficos como o ``pictex''
% Então é necessário usar apenas um dos pacotes conflitantes


% ---
% Informações de dados para CAPA e FOLHA DE ROSTO
% ---
\titulo{\emph{Sharpener}: Ferramenta de correção automática de programas para o ensino de programação com workflow expandido.}
\autor[Abbud, P. M.]{Pedro Morello Abbud}
\orientador[Orientador:]{Prof.\ Dr.}{Dilvan de Abreu Moreira}
\coorientador{Prof.\ Dr.}{Orlando de A.\ Figueiredo}
\curso{Engenharia da Computação}
\area{Sistemas Computacionais} % Área de concentração do trabalho
\data{02}{11}{2019} % Data do depósito
% ---


% ---
% RESUMOS
% ---

% Resumo em português
% conter no máximo 500 palavras
\textoresumo{
Propõe uma ferramenta (\emph{Sharpener}) de correção automática de soluções 
para exercícios de programação, que se diferencia por apoiar o aluno
estagnado na resolução de um exercício com várias opções de apoio: a) dicas, 
b) apresentação da resposta final do exercício 
(o que obriga o aluno a responder outro exercício equivalente), ou c) com
um roteiro de desenvolvimento incremental do exercício.
Um protótipo parcialmente funcional foi desenvolvido como prova de conceito.
A arquitetura do protótipo inclui: um servidor acessível a partir de 
uma API REST, compatível com computação em nuvem, com suporte a autenticação
federada, implementada com o \emph{framework} Flask (Python) e o SGBD Postgres; 
e uma aplicação cliente de linha de comando, implementada em Rust.
}{Ensino de programação, Sistemas Web}

% ---
% resumo em inglês
% ---
\textoresumo[english]{
Proposes an automatic programming exercises correction tool  (Sharpener), which is distinct for supporting students that are stuck in an exercise resolution, with aid options such as: a) hints; b) presentation of the final answer of the exercise, then making students answer another equivalent exercise; or c)  guided and incremental approach of the exercise.
A partially functional prototype was developed as a proof of concept. The architecture of the prototype includes a web server accessible via a REST API, compatible with Cloud Computing, that supports federated authentication and implemented using Python, with the Framework Flask, and the DBMS PostgreSQL; and a command-line interface, implemented in Rust.
}{Programming language education, \emph{Web} systems}
% ---
% Configurações de aparência do PDF final
% ---
% alterando o aspecto da cor azul
\definecolor{blue}{RGB}{41,5,195}

% informações do PDF
\makeatletter
\hypersetup{
     	pagebackref=true,
		pdftitle={\@title}, 
		pdfauthor={\@author},
    	pdfsubject={\imprimirpreambulo},
	    pdfcreator={LaTeX with abnTeX2/ICMC-USP},
		pdfkeywords={\palavraschave}, 
		colorlinks=true,       		% false: boxed links; true: colored links
    	linkcolor=blue,          	% color of internal links
    	citecolor=blue,        		% color of links to bibliography
    	filecolor=magenta,      	% color of file links
		urlcolor=blue,
		bookmarksdepth=4
}
\makeatother
% --- 

% ----------------------------------------------------------
% ELEMENTOS PRÉ-TEXTUAIS
% ----------------------------------------------------------

% Inserir a ficha catalográfica
% \incluifichacatalografica[tex/fichaCatalografica.pdf]
\incluifichacatalografica


% Inserir folha de aprovação
% \input{tex/folha-aprovacao}

% DEDICATÓRIA / AGRADECIMENTO / EPÍGRAFE
\textodedicatoria*{tex/pre-textual/dedicatoria}
\textoagradecimentos*{tex/pre-textual/agradecimentos}
% \textoepigrafe*{tex/pre-textual/epigrafe}

% Inclui a lista de figuras
\incluilistadefiguras

% Inclui a lista de tabelas
% \incluilistadetabelas

% Inclui a lista de quadros
% \incluilistadequadros

% Inclui a lista de algoritmos
% \incluilistadealgoritmos

% Inclui a lista de códigos
\incluilistadecodigos

% Inclui a lista de siglas e abreviaturas
% \incluilistadesiglas

% Inclui a lista de símbolos
% \incluilistadesimbolos

% ----
% Início do documento
% ----
\begin{document}

% ----------------------------------------------------------
% ELEMENTOS TEXTUAIS
% ----------------------------------------------------------
% \textual

\chapter{Introdução}
\label{chapter:introducao}
\section{Motivação e Contextualização}
Nas últimas décadas, o interesse por programação cresceu de forma extraordinária 
e cursos introdutórios mostram-se cada vez mais populares. No entanto, 
cursos de programação ainda são considerados difíceis e, com frequência,
possuem taxas de desistência elevada. Iniciantes sofrem de uma 
vasta gama de défices e dificuldades, desde a compreensão de constructos 
básicos de uma linguagem de programação à elaboração de planos
para resolver um dado problema \ \cite{robins2003learning}.

Estudos concluem, de forma semelhante, que alunos, 
mesmo tendo aprendido a sintaxe e semântica de uma linguagem 
de programação, tendem a falhar em compor os blocos fundamentais 
estudados em programas válidos. Parte deste problema é abordada por 
\citeonline{davies1993models}, que faz uma distinção de duas partes 
essenciais no aprendizado em programação: 
conhecimento e estratégia. Denota-se conhecimento em programação como a parte declarativa 
do conhecimento, por exemplo, saber que existe um constructo chamado \emph{for} na linguagem 
e seu propósito. Já estratégia em programação seria empregar de forma adequada o conhecimento 
em face de um problema, como o uso pertinente de um \emph{for} em um programa.
\citeonline{davies1993models} defende que grande parte das abordagens em ensino 
tendem a investir excessivamente na parcela de conhecimento em programação, muitas 
vezes deixando de lado o aspecto de estratégia.

Em vista do imenso desafio que é o ensino de programação a iniciantes, diversas ferramentas 
foram desenvolvidas para dar apoio pedagógico ao professor nesta tarefa.  
Estas ferramentas, em grande parte, são sistemas de juízes \emph{online}, 
que consistem em sistemas capazes 
de receber códigos desenvolvidos pelos usuários como resposta a um determinado problema, mas 
com a capacidade de dar \emph{feedback} imediato ao usuário, informando 
se o código está correto ou não, através de análise sintática e testes automatizados.

Já existe uma grande variedade em juízes \emph{online} disponíveis na literatura científica,
tais como: 
\hyperref[link:the_huxley]{\emph{The Huxley}} [\citeonline{de2013ferramenta}], 
	\hyperref[link:code_bench]{\emph{CodeBench}} [\citeonline{galvao2016juiz}],  
	\hyperref[link:uva_judge]{\emph{UVa Online Judge}} [\citeonline{revilla2008competitive}], 
	\hyperref[link:feeper]{\emph{feeper}} [\citeonline{alves2014ambiente}], 
	\hyperref[link:uri_judge]{\emph{URI Online Judge}} [\citeonline{bez2014uri}], 
	\hyperref[link:boca]{\emph{BOCA}} [\citeonline{de2004boca}],
	\hyperref[link:we_run_codes]{\emph{RunCodes}},
	\hyperref[link:sphere_judge]{\emph{Sphere Online Judge}},
	\hyperref[link:hacker_rank]{\emph{HackerRank}}, 
	\hyperref[link:code_chef]{\emph{CodeChef}}, 
	\hyperref[link:interview_bit]{\emph{InterviewBit}}, 
	\hyperref[link:kattis]{\emph{Kattis}}, 
	\hyperref[link:leet_code]{\emph{LeetCode}}, 
	\hyperref[link:codin_game]{\emph{CodinGame}}, 
	\hyperref[link:code_signal]{\emph{CodeSignal}}, 
	\hyperref[link:code_wars]{\emph{CodeWars}}, 
	\hyperref[link:exercism]{\emph{Exercism}}, 
	entre tantos outros.

Boa parte dessas ferramentas consiste em serviço gratuito oferecido ao aprendiz,
e com poucos recursos de acompanhamento por parte dos professores.
Outra parte, sim, é direcionada para situações acadêmicas, mas com 
\emph{workflows} de submissão e correção de programas rígidos;
qualquer intenção de personalizar uma típica ferramenta desse gênero
implica alterar o código-fonte.

O sistema que é tema do presente trabalho, o \emph{Sharpener}, surgiu do
anseio de se fazer pequenas adaptações a ferramentas de apoio a disciplinas
de programação.
Um problema observado é que alguns alunos não sabem como dar um próximo passo
na resolução de um exercício;
muitas vezes, eles precisam de um empurrão (como uma dica) para prosseguir.
Em aulas com muitos alunos, o professor nem sempre é capaz de proporcionar
esse tipo de acompanhamento e atenção, devido às limitações evidentes.
A questão passa a ser, então, se uma ferramenta poderia prover ao aluno
algum tipo de apoio, de forma que ele, seguindo a ferramenta, pudesse
explorar possibilidades e alternativas que o impulsionassem na continuação
do exercício; o aluno teria, com isso, uma relativa autonomia no aprendizado.

A solução mais imediata é capacitar a ferramenta a entregar dicas conforme
o aluno as solicita. Para isso, o professor precisaria ter alimentado o sistema
com dicas no mesmo momento em que teria carregado o enunciado do exercício.
Uma ideia similar é entregar ao aluno, de uma vez, a resposta completa do problema.
Muitos alunos têm uma ansiedade por ver respostas. 
A experiência de ver a resposta de um problema não resolvido pode ser um momento
de intensa aprendizagem.
Em vez de coibir isso, a ferramenta pode estimular, compensando com a 
apresentação, em seguida, de um exercício equivalente e alternativo.
Uma terceira proposta, ainda no espírito de fornecer algum tipo de dica,
é conduzir o aluno numa solução incremental do problemas, propondo pequenos
passos (incrementos) rumo à solução.
Com \emph{workflows} como esses, além do efeito pretendido de dar alguma 
autonomia ao aluno, há um efeito colateral que é a possibilidade de o professor
ter acesso ao histórico de consultas do aluno, podendo fazer uma avaliação
mais rica do seu percurso de aprendizagem.

Em vez de adaptar ferramentas existentes cujo código fosse aberto,
optou-se por criar um protótipo de uma ferramenta nova como prova de conceito.
O foco desse desenvolvimento passou a ser as componentes arquiteturais 
principais do sistema, como o servidor e sua API de acesso, e
uma ferramenta cliente de linha de comando.
Embora a contribuição pretendida aqui resida no domínio do
ensino de programação e suas ferramentas de apoio,
levando em conta que se trata de um Trabalho de Conclusão de Curso,
é inevitável tentar apresentar neste documento
alguns aspectos predominantes no desenvolvimento do protótipo, 
tais como: \emph{design} de APIs Web, estilo arquitetural \emph{REST}, computação em nuvem, autenticação federada,
e as principais decisões de projeto.
O escopo proposto inicialmente revelou-se excessivo, considerando-se o prazo de
um semestre letivo, e este trabalho passou a ser apenas a etapa inicial do
desenvolvimento do \emph{Sharpener}.

\section{Objetivos}
Pretende-se neste trabalho desenvolver um protótipo, como prova de conceito, de um sistema para apoio 
ao ensino de programação em salas de aula, que forneça apoio a estudantes estagnados no
desenvolvimento de exercícios, apoio esse que se manifesta
na forma de dicas, respostas, exercícios alternativos e
passos incrementais.

Em termos de software, pretende-se desenvolver minimamente uma arquitetura baseada em Web,
com um servidor acessível por \emph{API REST}, com suporte a autenticação federada, 
compatível com computação em nuvem, um cliente de linha de comando para funcionalidades
diretamente ligadas aos exercícios, e um cliente web para
funcionalidades administrativas básicas.

\section{Organização}
O cerne do trabalho é apresentado no \Cref{chapter:desenvolvimento}. Ali se encontram: a descrição
dos \emph{workflows} do principais atores (Aluno e Professor), seja na interface de linha de comando ou
na interface Web; e a descrição da API Web.

Dado o caráter de desenvolvimento de software do presente trabalho, diversos tópicos
foram mobilizados para a sua execução, entre metodologias, técnicas e ferramentas.
No \Aref{chapter:implementacao}, traça-se um panorama de todo o processo.
O \Cref{chapter:metodos} contém uma apresentação de alguns tópicos selecionados,
assim como o \Aref{chapter:cicd}.

Por fim, no \Cref{chapter:conclusao} estão as conclusões e contribuições do projeto, 
seguidas dos trabalhos futuros.


\chapter{Métodos, Técnicas e Tecnologias Utilizadas}
\label{chapter:metodos}
Este capítulo tem como propósito fornecer o embasamento teórico necessário 
para o entendimento da construção do sistema proposta. Aqui é contido uma
descrição detalhada das técnicas, métodos e tecnologias utilizadas, assim  
preparando o leitor para o conteúdo dos próximos capítulos.


\section{Design de APIs}
Separação de conceitos, traduzido do termo \emph{Separation of Concerns}, é uma temática 
chave na arquitetura cliente-servidor da internet. Parte do porquê a internet funciona tão bem
foi a preocupação, desde sua concepção, de uma interface uniforme que, desde que respeitada, 
 daria liberdade aos desenvolvedores de implementar, em qualquer linguagem ou tecnologia,
 um de seus componentes. 

Apesar do sucesso na separação de conceitos entre as responsabilidades que o navegador 
e o servidor empregam, a mesma preocupação não foi replicada entre interfaces e 
funcionalidades que o servidor expõe. 
Com frequência existe um alto acoplamento entre a interface, \emph{Frontend},
e o servidor, \emph{Backend}, o que resulta em uma interface não uniforme para interação 
dos recursos que deveriam estar expostos. Define-se aqui um recurso como qualquer conceito 
na internet que pode ser referenciado por um identificador único e manipulado por uma interface uniforme 
\ \cite{masse2011rest}.

A solução, que leva ao desacoplamento, é a interação com estas
funcionalidades través de uma \emph{API}, \emph{Application Programming Interface},
exposta pelo \emph{backend}, que provê formas padronizadas 
de acesso. Aqui não somente ganhamos portabilidade, já que possibilitamos que não só 
um tipo de cliente, navegadores, saibam como acessar nossos recursos, mas ganhamos espaço 
para criar uma camada de abstração do serviço \emph{Web}, modelando-o em recursos.
Estes recursos não serão projetados como uma cópia da organização de dados ou funcionalidades 
presentes no \emph{backend} mas em representações que sejam de fácil consumo e de entendimento 
ubíquo pelo lado do cliente.

O desafio de criar bons serviços na internet pode ser facilitado se empregarmos estilos 
arquiteturais já existentes. Aqui definimos estilos arquiteturais como 
um conjunto coordenado de restrições arquiteturais. \\
Um destes estilos arquiteturas que tem ganhado cada vez 
mais tração se chama \emph{REST}, \emph{Representational State Transfer}, ou em português,
\emph{Transferência Representacional de Estado}. Cunhado por \citeonline{fielding2000architectural}, 
o termo evoca como uma sistema na \emph{Web} deveria se comportar, uma máquina de estados 
virtuais em que o usuário progride através da seleção de identificadores únicos, que identificam 
recursos, e verbos \emph{HTTP}
que operam sobre estes recursos.

As restrições na arquitetura impostas pelo estilo \emph{REST} são agrupadas em seis categorias:
\begin{description}
\item[Cliente-servidor:] A separação dos papéis do cliente e servidor deve ser clara para 
  que estes possam ser projetados e implementados de forma independente. A interação 
  entre eles só acontece na forma de requisições, que são iniciadas pelo cliente. 
  Servidores devem mandar respostas apenas como reações de requisições dos clientes.
\item[Interface uniforme:] interfaces \emph{REST} possuem quatro  restrições:
identificação de recursos, manipulação de recursos através de representações, 
mensagens auto-descritivas e hipermídia como motor de estado da aplicação, \emph{HATEOAS}.\\
A primeira dita que cada recurso precisa ser enderessável 
por um identificador único, \emph{URI}, \emph{Unique Resource Identifier}.\\
Este 
recurso pode ser representado de diversas formas, seja um \emph{HTML}, que é mais adequado para 
um navegador,
ou \emph{JSON} que é mais apropriado para consumo de outro programa. Percebe-se aqui 
que a representação é apenas uma forma de interagir com o 
recurso,  não o próprio recurso. Isto é o que dita a segunda regra. \\
No consumo de um \emph{API}, o cliente especifica um recurso e seu estado 
desejado, enquanto o servidor deve responder com o recurso e seu estado real. 
Esta troca de mensagens  deve ser feita utilizando \emph{headers} e 
códigos de estado \emph{HTTP} que descrevam o estado 
do recurso e metadados correspondentes, de forma que as mensagens sejam auto-descritivas.\\
A última restrição, \emph{HATEOAS}, ajuda a aumentar a visibilidade e descoberta de recursos 
relacionados na \emph{API}, que ajudam o cliente a navegar dinamicamente nos recursos expostos.
Assim, quando referenciam-se outros recursos, também são fornecidos seus identificadores únicos 
e verbos aceitos para aquela rota.
\item[Sistema em camadas:] Restringe-se o comportamento dos componentes em camadas, em que 
  cada camada só pode interagir com subjacentes. 
  Entre as chamadas do cliente que requisita uma representação de um estado de um recurso, e 
  o servidor que processa a requisição, podem haver vários servidores entre eles. Estes 
  servidores podem prover um camada de segurança, \emph{cache}, balanceamento de carga ou 
  outras funcionalidades. Estas camadas não devem afetar a requisição ou resposta e nem o cliente 
  nem servidor precisam estar cientes se elas existem ou não.
\item[Protocolo sem estado:] O servidor não deve lembrar absolutamente nada do usuário que 
  utiliza sua \emph{API}. Isto implica que toda requisição individual precisa conter toda 
  informação necessária para que o servidor processe e retorne uma resposta. Esta restrição 
  visa aumentar a escalablidade, visibilidade e confiabilidade do sistema. Escalabilidade pois 
  permite que haja \emph{caching} de respostas e que o servidor possa desalocar recursos 
  físicos entre requisições, visibilidade pois sistemas que monitoram requisições não 
  precisam olhar além de uma requisição, pois elas contém todo o dado necessário 
  para entender a natureza desta, e a confibilidade pois a recuperação de falhas 
  parciais se torna mais simples \ \cite{kendall1994note}.
\item[Cache:] O servidor deve declarar quais dados podem ser guardados em \emph{caches}. 
  \emph{Caches} ajudam a reduzir a latência percebida pelo cliente, aumentam a disponibilidade 
  e confiabilidade de serviços e mitigam a carga de trabalho do servidor. Em resumo, 
  \emph{caches} reduzem todos os custos associados a um serviço na \emph{internet}.
\item[Código sob demanda:] A internet faz bastante uso de código sob demanda, que possibilita 
  que o servidor transfira executáveis, tais como \emph{scripts} e \emph{plugins}, para a 
  execução do lado do cliente. Apesar de proposto por \citeonline{fielding2000architectural}, 
  esta restrição tende a estabelecer um acoplamento de tecnologias entre servidor e cliente. 
  Por este motivo ``código sob demanda'' é a única restrição imposta pelo estilo arquitetural 
  \emph{REST} que é considerada opcional.
\end{description}

O maior desafio no projeto de uma \emph{API} é abstrair componentes do sistema
em recursos. O modelamento destes recursos estabelece os aspectos chaves da sua \emph{API}
e é semelhante ao processo de modelar um banco de dados relacional ou mesmo o modelamento 
clássico de um sistema orientado a objetos.

No processo de modelagem de recursos, rotineiramente começa-se pensando em arquétipos de 
recursos. Estes arquétipos nos ajudam a  comunicar de forma consistente 
as estruturas que são frequentemente encontradas em \emph{Designs} de \emph{REST APIs}.
Idealmente, cada recurso pertencerá a apenas um dos seguintes arquétipos: \emph{document}, 
\emph{collection}, \emph{controller} e \emph{store}.

\emph{Document} é o arquétipo base de todos os outros, todos os outros arquétipos são especializações 
destes. Sua representação tipicamente incluí campos com dados ou \emph{hiperlinks} para outros 
recursos. \emph{Collection} é um diretório de outros recursos em que clientes podem recuperar 
ou, possivelmente, adicionar novos items a coleção. Já \emph{Stores} são recursos que permitem 
que usuários guardem novos recursos, de forma que o próprio cliente escolha o 
identificador daquele recurso. Para recursos que se parecem mais como procedimentos que 
não se encaixam nos métodos padrões, \emph{CRUD} (\emph{create, retrieve, update, delete}),
  o arquétipo \emph{Controller} é adequado. Todos estes arquétipos 
  contém recomendações de como o identificador único deve ser escolhido, para que seja 
  claro a quem consome a API de qual arquétipo se trata aquele recurso.

  \section{O padrão \emph{OAuth}}
  Apesar de vivermos em um mundo que depende, cada vez mais, de serviços digitais,
  a maior parcela dos usuários ainda não se educaram ou 
  não seguem boas práticas de segurança na \emph{internet}. Não é infrequente que se 
  use as mesmas credenciais em múltiplos serviços, mesmo que o usuário 
  não tenha meios de averiguar que suas informações são manuseadas e armazenadas de forma 
  adequada.

  Uma das possíveis soluções para uma melhor gestão de identididade na \emph{internet}, 
  é o uso de identidades federadas \cite{208723}. Uma identidade federada é um meio 
  de uma pessoa vincular sua identidade eletrônica e atributos pessoais 
  a múltiplas partes distintas. Esta solução trás benefícios como mais garantias
  de segurança no armazenamento de credencias, menos sobrecarga de responsabilidade 
  para desenvolvedores de aplicações e uma melhor experiência para o usúario, já que se 
  diminuí o atrito na criação e gerenciamento de múltiplas contas.

  O modelo de serviço federado pode ser dividido em duas componentes necessárias: 
  o provedor de identidade e o provedor de serviço. Ambas as partes cumprem papéis 
  distintos e sem qualquer uma delas não temos uma solução de identidade federada. 
  Para que a solução funcione, deve se haver um relacionamento de confiança entre 
  o provedor de identidade e o provedor de serviço. Caso o provedor de identidade 
  não confie no provedor de serviço, este não o confiará os dados de usuários.
  Caso o provedor de serviço não confie no provedor de identidade, 
  também não confiará nas informações advindas dele.

  O provedor de identidade é quem mantém as informações de identidade. Usuários 
  se autenticam diante ao banco de credenciais do provedor, que então libera as
  informações relativas àquele usuário ao provedor de serviço que iniciou o processo 
  de autenticação. O provedor de serviço é aquele que provê serviços a usuários, 
  estes sendo aplicações, serviços de infraestrutura ou serviços de dados. Com a popularização 
  do modelo de computação em nuvem, cada vez mais usuários entram em contato com algum dos três 
  principais modelos de serviço em nuvem,  \emph{Infrastructure as a Service} (\emph{IaaS}), 
  \emph{Platform as a Service} \emph{PaaS} e \emph{SaaS}, \emph{Software as a Service}, 
  todas as quais depende de um modelo de serviço federado \cite{rountree2012federated}.

  Existem vários padrões que implementam o modelo de serviço federado, as mais famosas sendo:
  \emph{SAML} (Security Assertion Markup Language), \emph{OAuth}, \emph{OpenID}, 
  \emph{Simple Web Tokens}, \emph{JSON Web Tokens} e \emph{Windows Identity Foundation}. O 
  padrão que tem ganhado mais adeptos, pela sua abordagem simples porém robusta,
  é o padrão \emph{OAuth}, que já conta com quatro revisões, sendo a última
  \emph{OAuth 2.0}. 

  \emph{OAuth 2.0} é um padrão aberto criado para suportar autorização de forma segura. 
  Existem três papéis definidos pelo padrão: servidor de autorização, servidor de recurso e 
  dono do recurso. Em modelos tradicionais cliente-servidor, o dono do recurso, ou seja, cliente 
  acessaria recursos no servidor de recurso, desde que este consiga se autenticar com 
  sucesso no servidor. No padrão \emph{OAuth} temos duas ou três partes envolvidas, 
  sendo duas quando o servidor de autorização também é o dono do recurso.

  \emph{OAuth 2.0} suporta vários fluxos de autenticação: fluxo usuário-agente, 
  fluxo servidor \emph{Web}, fluxo de dispositivo, fluxo de usuário e senha, fluxo 
  de credencias de cliente e fluxo de asserção. A maioria das aplicações \emph{Web}, 
  principalmente quando se tratam de \emph{Single Page Applications},
  fazem uso do fluxo servidor \emph{Web}, também 
  conhecido por ``concessão de código de acesso" (\emph{Authorization Code Grant})
  que é ilustrado pela \fref{fig:web_flow} e definido por \citeonline{hardt2012rfc}. 

  O fluxo de concessão de código de acesso começa quando um usuário, dono de um recurso, escolhe 
  fazer \emph{login} ou atrelar sua conta do servidor de autorização em uma aplicação. A aplicação 
  faz uma requisição ao servidor de autorização que por sua vez retorna uma página de \emph{login}. 
  O usuário preenche suas credencias na página e autoriza a aplicação a acessar um determinado 
  escopo de recursos. Em seguida, o servidor de autorização emite um código de autorização, 
  que geralmente expira em minutos. A aplicação, em posse do código de autorização, requisita 
  ao servidor de autorização um código de acesso. Com o código de acesso, a aplicação agora 
  pode acessar recursos protegidos do servidor de recursos, até que este código expire.

  Existem alguns detalhes que garantem a segurança do fluxo anterior. Quando a aplicação requisita 
  o servidor de autorização por um código de autorização, ela informa um identificador de cliente, que 
  indica sua identididade, uma \emph{url} de retorno, que é para aonde o servidor de autorização mandará 
  o código de acesso e uma cadeia de caracteres aleátorios que será usado para validar se a requisição 
  veio do servidor de autorização. Caso na requisição a \emph{url} de retorno informada seja diferente 
  da cadastrada pela aplicação no servidor de autorização previamente, a requisição é abortada. No 
  recebimento da resposta do servidor de autenticação, caso a cadeia de caracteres aleátorios não seja a mesma 
  da enviada, aborta-se o processo. Para obtenção do código de acesso, a aplicação também não somente deve informar 
  ao servidor de autenticação o código de autorização, mas também um segredo previamente cadastrado em conjunto 
  com seu identificador.
  \begin{figure}[htpb]
    \centering
    \includegraphics[width=0.95\linewidth]{images/web_flow.png}
    \caption{Diagrama que explica o funcionamento do fluxo servidor \emph{Web}, 
      definido na seção 4.1 da norma \emph{RFC 6749}, retirado de \url{https://www.authlete.com}.}%
    \label{fig:web_flow}
  \end{figure}
  

  \section{Integração e entrega contínua}
  Um dos grandes desafios no desenvolvimento de \emph{software} é criar um processo repetível e 
  confiável para entregas de aplicações. Frequentemente, lançamentos de \emph{software} são
  tratados por seus desenvolvedores como momentos estressantes. Associa-se este \emph{stress} a
  muitos processos manuais, infrequentes e propensos a erros, conduzidos em curtos prazos 
  de tempo.
  
  Pode-se, no entanto, se um processo rigoroso for seguido,  
  tornar esta tarefa fácil, tão fácil quanto o apertar de um botão.
  Para atingirmos este nível de maturidade no desenvolvimento de 
  \emph{software}, \citeonline{humble2010continuous} propõe que
  devemos seguir a risca dois princípios: automatização de todas etapas que sejam 
  automatizáveis e manter todos os artefatos necessários para um lançamento em um sistema de 
  controle de versões. 

  Idealmente, uma entrega de \emph{software} é composta por três coisas: 
  provisionar e gerenciar um ambiente que a aplicação irá rodar (configuração de \emph{hardware},
  \emph{software}, infraestrutura e serviços externos), instalar a versão correta da aplicação 
  neste ambiente e configurar a aplicação, incluindo qualquer estado ou dado que ela possa requerer.

  De fato, até recentemente, vários destes passos para a entrega de \emph{software} pareciam 
  impossíveis ou ao menos complexas de aderirem os princípios citados por 
  \citeonline{humble2010continuous}. Como poderíamos, por exemplo, 
  versionar \emph{hardware}? Com o advento de virtualização eficiente e barata, até esta 
  tarefa que, \emph{a priori}, parecia impossível, virou algo corriqueiro e ubíquo. 
  Com advento e adoção da computação em nuvem, todos os passos necessários para automatização 
  de testes, compilação ou de artefatos e lançamento do \emph{software} estão 
  acessíveis a qualquer desenvolvedor.

  Define-se integração contínua como o processo no desenvolvimento de \emph{software} em que 
  membros de um time integram seu trabalho de forma frequente. Cada integração desencadeia etapas 
  como verificação de estilo de código, testes automatizados e construções de compilados ou 
  outros artefatos, com o objetivo de encontrar problemas nas integrações o mais rápido possível.
  Já entrega contínua leva a prática de integração contínua para outro patamar e automatiza 
  o lançamento de versões do \emph{software} dado que as etapas anteriores foram concluídas com 
  sucesso e o código foi revisado \cite{fowler2006continuous}.


%\chapter{Desenvolvimento}
%\label{chapter:desenvolvimento}
%\section{Considerações Iniciais}
Este capítulo apresenta o projeto como um todo, explicando escolhas que foram 
feitas em seu desenvolvimento, tais como ferramentas e tecnologias adotadas. 
 


\section{Projeto}
Com o intuito de desenvolver um ambiente de ensino que fosse igualmente conveniente 
para professores estruturarem suas disciplinas de programação, 
tanto quanto agradável para alunos participarem de atividades, desenvolveu-se nesta monografia
 uma prova de conceito do que se envisinou poder ser esse sistema. Espera-se que, a partir 
 desta prova de conceito, atraía-se interessados pela construção de uma solução e 
 que, de forma coletiva, construa-se uma plataforma de excelência,
 por meio de colaborações em código aberto.

O professor na plataforma é capaz de criar e gerenciar exercícios, trilhas e turmas. 
Alunos se inscrevem na plataforma por meio de um \emph{login} de identidade federada, e 
ingressam nas turmas por meio de um \emph{link} específico, que é fornecido 
pelo professor nos primeiros dias de aula. 

Trilhas de exercícios são preparadas por professores, que a associam às suas turmas. 
Estas trilhas são compostas de Grupos de Exercícios Equivalentes que 
contêm exercícios de assuntos correlatos e, idealmente, de mesmo nível de dificuldade.
Para cada aluno, é sorteado um exercício de cada grupo de forma aleatória. 
Caso o aluno apresente dificuldade na resolução deste exercício, o aluno pode solicitar 
dicas relativas ao exercício, ou, em último caso, a solução do mesmo. Caso a solução 
seja requisitada, outro exercício do mesmo passo é sorteado a este aluno.

Para diminuir o escopo inicial do projeto e incentivar o estudante a se familiarizar com o ferramental 
que envolve desenvolvimento de software, escolheu-se que o desenvolvimento do código 
acontecesse localmente no computador do aluno. Para que, ainda sim, proporcionemos ao aluno 
uma boa experiência, desenvolveu-se uma interface por linha de comando, que nos referiremos a partir de agora 
por \emph{CLI} (\emph{Command Line Interface}). Por meio desta \emph{CLI},
o aluno pode fazer \emph{downloads} e submissões
dos exercícios propostos de maneira rápida e prática.

Dessa forma o sistema está dividido em três partes: uma interface gráfica, \emph{frontend}, 
em que professores e alunos possam interagir com exercícios, trilhas e turmas, uma \emph{CLI} que possibilita 
\emph{download} e submissão de exercícios, e um servidor com uma \emph{API} capaz de abstrair 
os recursos necessários por ambos os clientes. Na \fref{fig:arquitetura} apresenta-se um diagrama 
da arquitetura do sistema.

  \begin{figure}[htpb]
    \centering
    \includegraphics[width=\linewidth]{images/arquitetura.pdf}
    \caption{Diagrama que explica a arquietura do sistema \emph{Sharpener}, o qual 
    se desenvolveu uma prova de conceito.}%
    \label{fig:arquitetura}
  \end{figure}



\section{Atividades Realizadas}
\subsection{Desenvolvimento da \emph{API}}
Para o desenvolvimento da \emph{API}, escolheu-se a linguagem \emph{Python}, por sua clareza, 
e sua alta produtividade, que é crucial no processo de prototipação. Em conjunto com \emph{Python}, 
o \emph{micro framework web} \emph{Flask} também foi escolhido. Justifica-se essa escolha pela
inerente simplicidade, clareza e produtividade do \emph{framework}. 

Antes de podermos modelar recursos para a \emph{API},
estudou-se quais dados eram necessários para os casos de uso propostos. Este estudo gerou um 
\emph{MER}, modelo entidade relacional, que relacionava todas as entidades do sistema. A partir 
do diagrama gerado, criou-se ``\emph{models}'' que representavam estas entidades. A classe 
que possibilitou a criação destes \emph{models} provêm da \emph{ORM} que abstraí bancos 
relacionais chamada \hyperref[link:sqlalchemy]{\emph{SQLAlchemy}}. \hyperref[link:sqlalchemy]{\emph{SQLAlchemy}}
é uma biblioteca estável com mais de treze anos de maturidade,
mas que continua sendo a escolha padrão dos desenvolvedores.

Optou-se por utilizar \hyperref[link:postgresql]{\emph{PostgreSQL}}.
A escolha foi feita por se tratar de um sistema 
gerenciador de banco de dados relacional de código aberto e por este ser referência em 
desempenho e funcionalidades.

No mapeamento do \emph{MER}, previamente esboçado, em classes utilizou-se a biblioteca 
de visualização \hyperref[link:eralchemy]{\emph{ERAlchemy}}. \hyperref[link:eralchemy]{\emph{ERAlchemy}} é capaz de gerar 
diagramas \emph{MER} de forma automática a partir das classes modelo ou de uma conexão com o banco de dados. 
Tal ferramenta se provou excepcionalmente útil pois permitiu o desenvolvimento incremental e iterativo das 
classes modelo e garantiu que o modelo conceitual fosse implementado sem divergências. A \fref{fig:mer} mostra 
o modelo entidade relacional do sistema, gerado a partir da ferramenta.

Para que nosso sistema fosse capaz de armazenar submissões de alunos de forma confiável,  
um serviço de intervalos de  armazenamento, traduzido do inglês \emph{Bucket Storage}, se 
fez necessário. Um intervalo de armazenamento nada mais é que uma abstração de provedores 
de computação em nuvem para oferecer armazenamento de objetos. Uma das grandes vantagens associadas 
a este tipo de serviço é o baixo custo por \emph{gigabyte}, além de sua alta escalabilidade, tanto 
 um aumento no volume de dados armazenados, quanto na velocidade recuperação destes.
O provedor de computação em nuvem escolhido foi a \hyperref[link:gcp]{\emph{Google Cloud Plataform}}.


A alternativa a adotar um serviço de armazenamento por intervalos seria armazenar arquivos 
diretamente no banco de dados, o que traria um impacto na performance do mesmo, já que \emph{SGBD}s não são otimizados para
este tipo de dado. 

Modelou-se recursos seguindo o estilo arquitetural \emph{REST}. Coleções e documentos
de exercícios, turmas, tópicos, entre outros recursos foram implementados, e um 
recurso do arquétipo \emph{controller} foi necessário para prover autenticação a interface por 
meio do fluxo de concessão do código de autorização. Também disponibilizou-se uma rota para \emph{health check}, 
que permite que uma sonda externa consulte se a aplicação continua funcionando adequadamente. Caso esta não esteja, 
envia-se um sinal para que um novo container da aplicação seja iniciado e o tráfego é redirecionado a ela. 

Para que não seja necessário que professores criem um banco de exercícios do zero, um ``conector'' foi 
implementado para a plataforma \emph{exercism}, em que seus exercícios são extraídos e guardados no banco 
de dados. A implementação aborda apenas duas linguagens \emph{Python} e \emph{Rust}, mas 
pode ser facilmente estendido para outras linguagens, dado que se informe a estrutura de arquivos 
que os exercícios daquela linguagem são armazenados.

Também foi configurado uma plataforma de \emph{CI/CD} para o projeto. A toda nova versão enviada 
ao repositório remoto do sistema de controle de versões, uma tarefa rodava a ferramenta \emph{autopep8}
que checava se o código \emph{commitado} era sintaticamente válido e não seguia más práticas. 
Caso o código fosse reprovado na tarefa anterior, este não poderia ser aprovado e mesclado 
na \emph{branch} principal de desenvolvimento. Se a contribuição passar na verificação e 
depois de revisão for aceita, uma tarefa para lançamento da nova versão é disparada automaticamente, 
       que coloca uma nova versão no ar. A plataforma de \emph{CI/CD} empregada foi o 
       \hyperref[link:actions]{\emph{Github Actions}} e 
       as novas versões eram colocadas no ar através da \emph{Platform as a Service} do \emph{Google}, 
       \hyperref[link:appengine]{\emph{AppEngine}}.

\begin{sidewaysfigure}[htpb]
    \centering
    \includegraphics[width=\linewidth]{images/db_schema}
    \caption{Modelo entidade relacional gerado a partir da ferramenta \hyperref[link:eralchemy]{\emph{ERAlchemy}}.}\label{fig:mer}
\end{sidewaysfigure}


\subsection{Desenvolvimento da interface}
A interface foi desenvolvida utilizando \emph{Javascript}, \emph{CSS} e \emph{HTML}. Como 
desejávamos uma plataforma que fosse bastante interativa, uma \emph{single page application} 
foi implementada, utilizando a tecnologia \hyperref[link:react]{\emph{React}}.
Seguiu-se o \emph{design system} chamado 
\emph{Material Design} por ser bastante intuitivo para novos usuários. A biblioteca \emph{Material-UI} 
forneceu vários componentes que serviram de base para os componentes customizados. A comunicação com a 
\emph{API} foi feita utilizando o cliente \emph{HTTP} \hyperref[link:axios]{\emph{axios}} e 
os dados são persistidos numa \emph{store} local, utilizando o gerenciador de estados \emph{Redux}.


\subsection{Desenvolvimento da CLI}
O desenvolvimento da \emph{CLI} foi feito na linguagem \emph{Rust}, utilizando 
a biblioteca \emph{StructOpt}. \emph{StructOpt} é uma biblioteca que permite construção 
de interfaces por linhas de comando a partir da anotação de macros em \emph{structs} ou 
\emph{enums}. Com uma simples anotação ganha-se um \emph{parser} dos comandos e mensagens 
que instruem o usuário como utilizar sua \emph{CLI}.
 
 Aceitam-se três comandos na \emph{CLI}, ``\emph{download}'' que a partir de um identificador busca 
 o exercício proposto, \emph{submit} que manda uma tentiva de solução do problema ao servidor e 
 \emph{config}, necessário que seja executado, ao menos uma vez, com uma chave que identifica 
 qual aluno está utilizando o programa.

\section{Resultados}
A \fref{fig:login} mostra a página inicial da interface \emph{web} da ferramenta \emph{Sharpener}.
O \emph{login} do usuário é feito por contas previamente cadastradas na plataforma \emph{Github},
por meio fluxo de código de acesso. O professor quando logado, será direcionado a página 
de suas turmas, retratada na \fref{fig:turmas}, em que poderá gerenciá-las. Nesta página 
o professor também inscreve suas turmas em trilhas previamente criadas, como podemos 
observar em \fref{fig:enroll_track}. 

Pode-se acessar a página que possibilita a criação de trilhas 
através do menu lateral, que o leva a página retratada pela \fref{fig:track}. Nesta página 
pode-se criar novas trilhas como mostrado na \fref{fig:add_track1}. Uma trilha é composta por 
vários ``passos'', que são os \emph{clusters} de exercícios discutidos anteriormente. 
Ao clicar no botão de adicionar exercícios, surge um novo componente em que pode-se buscar 
e selecionar exercícios que farão parte daquele passo, conforme podemos observar em 
\fref{fig:add_track4}. Sabemos que a elaboração de exercícios é uma tarefa que toma 
muito tempo, portanto na página de exercícios, retratada pela \fref{fig:exercicios},
professores podem buscar por novos exercícios,  que foram criados por outros professores 
ou extraídos de alguma repositório público.


\section{Dificuldades e Limitações}
Encontraram-se duas grandes dificuldade na condução deste trabalho. 
A primeira delas foi a orquestração de diferentes tecnologias e partes do sistema. Houve um 
alto custo associado a dominar, utilizar e integrar diferentes linguagens, \emph{frameworks} 
e serviços. Ao longo do desenvolvimento deste projeto, percebeu-se que o escopo do projeto 
abordado foi inadequado para desenvolvimento no decorrer de um semestre.

Apesar, desde a concepção do projeto, o desenvolvimento proposto era de apenas uma prova 
de conceito de um sistema, esperava-se que, ao final da disciplina, um protótipo já adequado 
para testes em salas de aula fosse alcançado. Infelizmente, tal expectativa não foi cumprida.  
Apesar de uma parte significativa do sistema ter sido contemplada, ainda faltam 
muitas funcionalidades essenciais para que um ``teste de campo'' seja realizado.
% -> Login 
% -> Criação de turmas
% -> Criação de tracks
% -> Cluster de exercícios

  \begin{figure}[htpb]
    \centering
    \includegraphics[width=\linewidth]{images/mocks/login.png}
    \caption{Página de \emph{Login} da prova de conceito do sistema \emph{Sharpener}.}%
    \label{fig:login}
  \end{figure}

  \begin{figure}[htpb]
    \centering
    \includegraphics[width=\linewidth]{images/mocks/turma.png}
    \caption{Página de turmas da prova de conceito do sistema \emph{Sharpener}.}%
    \label{fig:turmas}
  \end{figure}

  \begin{figure}[htpb]
    \centering
    \includegraphics[width=\linewidth]{images/mocks/turmaAddTrackSearch.png}
    \caption{Página de turmas da prova de conceito do sistema \emph{Sharpener}, em 
	    que o professor inscreve suas turmas em trilhas.}%
    \label{fig:enroll_track}
  \end{figure}

  \begin{figure}[htpb]
  \centering
  \includegraphics[width=\linewidth]{images/mocks/track.png}
  \caption{Página de trilhas da prova de conceito do sistema \emph{Sharpener}.}%
  \label{fig:track}
  \end{figure}

  \begin{figure}[htpb]
  \centering
  \includegraphics[width=\linewidth]{images/mocks/trackAdd1.png}
  \caption{Página de trilhas da prova de conceito do sistema \emph{Sharpener}, 
  em que uma nova trilha é criada.}%
  \label{fig:add_track1}
  \end{figure}

  \begin{figure}[htpb]
  \centering
  \includegraphics[width=\linewidth]{images/mocks/trackAdd4.png}
  \caption{Página de trilhas da prova de conceito do sistema \emph{Sharpener}, 
	  em que \emph{clusters de exercícios} são associados a passos de uma trilha.}%
  \label{fig:add_track4}
  \end{figure}

\begin{figure}[htpb]
  \centering
  \includegraphics[width=\linewidth]{images/mocks/exercicios.png}
  \caption{Página de exercícios da prova de conceito do sistema \emph{Sharpener}.}%
  \label{fig:exercicios}
  \end{figure}


\chapter{Desenvolvimento}
\label{chapter:desenvolvimento2}
\section{Considerações Iniciais}

\section{Visão Geral}

% Quem são os atores do sistema? R: Professor e aluno

% Quais são os objetivos do professor no sistema? recuperar o plano 
% no Google Docs como redação (menos as partes que desistimos de fazer
Do ponto de vista do professor, traz: 1) geração aleatória de exercícios a partir de templates; 2) anotação em respostas (especialmente de erros) para formação de um banco de exercícios; 3) geração de exercícios com programas parcialmente resolvidos (para desenvolvimento incremental); 4) geração de exercícios com programas com erros (para depuração); 5) acompanhamento do aluno levando em conta as escolhas de tipos de exercícios. Propõe desenvolver um protótipo como prova de conceito.
% Quais são os objetivos do aluno no sistema?
% recuperar plano, idem
O professor na plataforma é capaz de criar e gerenciar exercícios, trilhas e turmas. 
Alunos se inscrevem na plataforma por meio de um \emph{login} de identidade federada, e 
ingressam nas turmas por meio de um \emph{link} específico, que é fornecido 
pelo professor nos primeiros dias de aula. 

Trilhas de exercícios são preparadas por professores, que a associam às suas turmas. 
Estas trilhas são compostas de Grupos de Exercícios Equivalentes que 
contêm exercícios de assuntos correlatos e, idealmente, de mesmo nível de dificuldade.
Para cada aluno, é sorteado um exercício de cada grupo de forma aleatória. 
Caso o aluno apresente dificuldade na resolução deste exercício, o aluno pode solicitar 
dicas relativas ao exercício, ou, em último caso, a solução do mesmo. Caso a solução 
seja requisitada, outro exercício do mesmo passo é sorteado a este aluno.

Para diminuir o escopo inicial do projeto e incentivar o estudante a se familiarizar com o ferramental 
que envolve desenvolvimento de software, escolheu-se que o desenvolvimento do código 
acontecesse localmente no computador do aluno. Para que, ainda sim, proporcionemos ao aluno 
uma boa experiência, desenvolveu-se uma interface por linha de comando, que nos referiremos a partir de agora 
por \emph{CLI} (\emph{Command Line Interface}). Por meio desta \emph{CLI},
o aluno pode fazer \emph{downloads} e submissões
dos exercícios propostos de maneira rápida e prática.

Dessa forma o sistema está dividido em três partes: uma interface gráfica, \emph{frontend}, 
em que professores e alunos possam interagir com exercícios, trilhas e turmas, uma \emph{CLI} que possibilita 
\emph{download} e submissão de exercícios, e um servidor com uma \emph{API} capaz de abstrair 
os recursos necessários por ambos os clientes. Na \fref{fig:arquitetura} apresenta-se um diagrama 
da arquitetura do sistema.

  \begin{figure}[htpb]
    \centering
    \includegraphics[width=\linewidth]{images/arquitetura.pdf}
    \caption{Diagrama que explica a arquietura do sistema \emph{Sharpener}, o qual 
    se desenvolveu uma prova de conceito.}%
    \label{fig:arquitetura}
  \end{figure}
% Arquitetura resumida: a figura que já está pronta e o texto lá.

% Implementação, um resumo das linguagens e ferramentas 

\section{O sistema do ponto de vista do Aluno}

A rotina do Aluno no \emph{Sharpener} consiste em: ter acesso a um exercício, recorrer a alguma forma de apoio, completar o exercício submetendo uma versão, obter \emph{feedback}, e acompanhar seu andamento no curso. As principais atividades são realizadas na \emph{CLI}.

\subsection{Configuração}
O aluno tem acesso a uma interface de linha de comando, 
disponível para \emph{download} no \emph{Frontend}. Em seu primeiro 
acesso, o Aluno deve declarar que é o dono de uma \emph{token} que o identifica, isto é feito através do comando descrito em \cref{codigo:config}. Também deve ingressar na turma referente a seu professor, através de um \emph{link}, único para cada turma,
que será disponibilizado em sala de aula.

\begin{codigo}[caption = {Configuração da \emph{CLI}.}, label={codigo:config},language=bash, breaklines=true, mathescape=true]
student@usp:~$ sharpener config <identificador-aluno>
\end{codigo}

\subsection{Acesso a exercícios}
Com a \emph{CLI} configurada, o aluno escolhe
no \emph{frontend} um dos exercícios disponíveis e copia um identificador deste exercício. Com 
o identificador em mãos, o comando descrito em \cref{codigo:download} é utilizado para criar 
uma pasta com o nome do exercício no diretório que o comando foi invocado. Na pasta todos os 
arquivos necessários para o desenvolvimento de uma solução estão presentes: descrição detalhada 
do problema, testes, um arquivo especificando dependências e um ponto de entrada.
Cada linguagem apresenta uma estrutura diferente de pastas, o \cref{codigo:pasta} mostra a execução 
do comando ``\emph{tree}'' que mostra como é a organização dos arquivos para a linguagem \emph{Rust}, no exercício \emph{accumulate}.
O aluno deve desenvolver sua solução no arquivo de ponto de entrada, 
que no exemplo anterior, se encontra em ``src/lib.rs''.

\begin{codigo}[caption = {Download de exercício pela \emph{CLI}.}, label={codigo:download},language=bash, breaklines=true]
student@usp:~$ sharpener download <identificador-exercicio>
\end{codigo}

\begin{codigo}[caption = {Estrutura de pasta para um exercício em Rust}, label={codigo:pasta},language=bash, breaklines=true]
student@usp~$ tree
    Cargo.toml
    README.md
    src
        lib.rs
    tests
        accumulate.rs
2 directories, 5 files

\end{codigo}

% explicar a consulta na CLI (não existe, vamos desenhar/projetar)
% explicar o download na CLI

\subsection{Apoio a resolução de exercício}
Para que o aluno consiga evoluir sua solução até uma que esteja correta, testes automatizados estão presentes 
nos exercícios disponíveis no \emph{Sharpener}. Com o comando descrito em \cref{codigo:test}, os testes são 
executados e impressos no terminal. Caso todos os testes forem cumpridos o código está pronto para submissão, do contrário 
mudanças devem ser feitas no código e repete-se o processo.

\begin{codigo}[caption = {Executando a bateria de testes a partir da \emph{CLI}.}, label={codigo:test},language=bash, breaklines=true]
student@usp:~$ sharpener test 
\end{codigo}

% Buscando dicas
Caso o professor deseje disponibilizar dicas a cerca do exercício, o Aluno pode invocar o comando \cref{codigo:hint}. 
Uma dica será mostrado no terminal, guiando o Aluno na resolução do exercício.

\begin{codigo}[caption = {Comando para solicitar dicas do exercício na \emph{CLI}.}, label={codigo:hint},language=bash, breaklines=true]
student@usp:~$ sharpener hint
\end{codigo}

% Buscando/obtendo a resposta final
O Aluno, em último caso, pode desistir de resolver aquele exercício em específico do Grupo de Exercícios Equivalentes. 
O comando descrito em \cref{codigo:solution}, após uma confirmação do aluno, faz \emph{download} de um arquivo contendo uma possível 
solução. Quando o comando é emitido, é registrado que aquele exercício do Grupo de Exercícios Equivalentes foi pulado e um novo exercício é 
baixado e colocado no diretório pai do exercício atual.
\begin{codigo}[caption = {Comando para requisitar a solução de um exercício na \emph{CLI}}, label={codigo:solution},language=bash, breaklines=true]
student@usp:~$ sharpener solution
\end{codigo}


% Buscando/obtendo um Exercício Equivalente/Alternativo

\subsection{Submissão de resposta}
% 
Se a solução desenvolvida pelo Aluno cumprir todos os testes, sua submissão é possível. Com o comando descrito por \cref{codigo:submit}, 
sua solução é enviada, juntamente com o \emph{output} dos testes. Caso o comando seja invocado e os testes ainda não estão sendo cumpridos, um 
alerta notifica o Aluno que este é o caso, e o questiona se deseja continuar mesmo assim. Múltiplas tentativas de submissão podem ser feitas, 
sendo todas elas salvas pelo sistema. Assume-se que a última submissão é a que deve ser corrigida, mas o professor, se desejar, pode inspecionar tentativas feitas anteriormente.

\begin{codigo}[caption = {Comando para submeter exercícios na \emph{CLI}}, label={codigo:submit},language=bash, breaklines=true]
student@usp:~$ sharpener submit
\end{codigo}

\subsection{Exercícios incrementais}

\subsection{Acompanhamento do andamento na disciplina}

\section{O sistema do ponto de vista do Professor} 
\subsection{Gerenciando turmas e trilhas}
A jornada do professor também começa no \emph{Frontend}, que tem sua página inicial descrita pela \fref{fig:login}.
Após seu \emph{login} pelo \emph{Github}, este deve informar a um administrador 
do sistema que sua conta precisa ser elevada a condição de professor, já que todas as contas são alunos por \emph{default}.

  \begin{figure}[htpb]
    \centering
    \includegraphics[width=\linewidth]{images/mocks/login.png}
    \caption{Página de \emph{Login} da prova de conceito do sistema \emph{Sharpener}.}%
    \label{fig:login}
  \end{figure}

Assim que o Professor fizer \emph{login} na plataforma, o Professor é direcionado para página de turmas, como podemos visualizar na \fref{fig:turmas}. 
Este deve criar sua primeira turma, a dando um nome único, como mostra a \fref{fig:turmasAdd}. 

O próximo passo é a criação de uma trilha de exercícios, que pode ser feita na página descrita pela \fref{fig:track}. Cada trilha
contém Grupos de Exercícios Equivalentes, que no \emph{frontend} foram apelidados como ``grupos''. O professor associa exercícios 
a cada um destes grupos, podendo filtrá-los por nome na barra superior direto. 
Tal sequência é mostrada pela \fref{fig:add_track1}, \fref{fig:add_track2}, \fref{fig:add_track3} e \fref{fig:add_track4}.  

Caso o professor ainda não tenha definido todos os exercícios que compõe sua trilha, uma visita a página de exercícios pode lhe 
ser útil. A página conta com todos os exercícios disponíveis no banco de exercícios que podem ser filtrados por nomes ou tópicos, 
como pode ser visto pela \fref{fig:exercicios}.

O passo final é associar a trilha recém criada às suas turmas. Isto pode ser feito voltando na página de turmas, selecionando 
turmas e clicando no botão ``Adicionar \emph{Track}''. Um modal cobre a tela e é possível selecionar e procurar trilhas, dentre 
as disponíveis, para associar a estas turmas. A \fref{fig:enroll_track1} e \fref{fig:enroll_track2} mostram esta interação.

% Criar uma turma |||  TELAS, TELAS, TELAs, muitas telas
 





% Preparar e carregar exercícios
\subsection{Preparando exercícios}
Para a preparação de novos exercícios que farão parte do banco de exercícios, o professor precisa
ter baixado, instalado e configurado a \emph{CLI}, da mesma forma descrita anteriormente. O comando para criar um novo exercício o Professor deve invocar o comando com a sintaxe descrita em \cref{codigo:new}. Este comando criará um diretório novo diretório com os arquivos 
que deverão ser preenchidos pelo professor antes de submeter o exercício para o banco 
de exercícios. Os arquivos a serem preenchidos são: um arquivo chamado ``README.md'', contendo as instruções 
do que se espera no exercício, um arquivo contento testes, outro arquivo que conterá código produzido pelo aluno 
e um arquivo que contém uma solução para o exercício, escrito pelo professor.

\begin{codigo}[caption = {Comando para criar novos exercícios na \emph{CLI}}, label={codigo:new},language=bash, breaklines=true]
prof@usp:~$ sharpener exercise new <language> <exercise_name>
\end{codigo}

Se o Professor desejar adicionar dicas para o exercício, este deve preencher o arquivo escondido,
``.meta.json''. A chave \emph{``hints''} deve conter um \emph{Array} de \emph{Strings}, em que 
cada uma delas é uma dica. O campo ``topics'' deste mesmo arquivo pode ser preenchido para relacionar 
este exercício a estes tópicos.

Para submeter o exercício criado o professor deve executar o comando descrito por \cref{codigo:post} na 
pasta que contém o exercício.

\begin{codigo}[caption = {Comando para criar novos exercícios na \emph{CLI}}, label={codigo:post},language=bash, breaklines=true]
prof@usp:~$ sharpener exercise post
\end{codigo}

Caso deseje alterar um exercício, poderá baixá-lo através do comando \cref{codigo:download_prof} e 
resubmeté-lo pelo \cref{codigo:post}. Apenas professores estão autorizados a utilizar o 
\cref{codigo:download_prof} e resubmissões só podem ser feitas por exercícios de autoria daquele 
professor.
\begin{codigo}[caption = {Comando para criar novos exercícios na \emph{CLI}}, label={codigo:download_prof},language=bash, breaklines=true]
prof@usp:~$ sharpener exercise download <language> <exercise_name>
\end{codigo}

%% Enunciados
%% Casos de teste
%% Dicas
%% Grupo de Exercícios Equivalentes/Alternativos
%% Exercícios incrementais

% Acompanhar uma turma ou um aluno




 \begin{figure}[htpb]
    \centering
    \includegraphics[width=\linewidth]{images/mocks/turmaExpandido.png}
    \caption{Página de turmas da prova de conceito do sistema \emph{Sharpener}.}%
    \label{fig:turmas}
  \end{figure}
  
  \begin{figure}[htpb]
    \centering
    \includegraphics[width=\linewidth]{images/mocks/turmaNova.png}
    \caption{Página de turmas da prova de conceito do sistema \emph{Sharpener}, em que se 
    cria uma nova turma.}%
    \label{fig:turmasAdd}
  \end{figure}

  \begin{figure}[htpb]
  \centering
  \includegraphics[width=\linewidth]{images/mocks/track.png}
  \caption{Página de trilhas da prova de conceito do sistema \emph{Sharpener}.}%
  \label{fig:track}
  \end{figure}
  
  \begin{figure}[htpb]
  \centering
  \includegraphics[width=\linewidth]{images/mocks/trackAdd1.png}
  \caption{Página de trilhas da prova de conceito do sistema \emph{Sharpener}, 
  em que uma nova trilha é criada.}%
  \label{fig:add_track1}
  \end{figure}
  
    \begin{figure}[htpb]
  \centering
  \includegraphics[width=\linewidth]{images/mocks/trackAdd2.png}
  \caption{Página de trilhas da prova de conceito do sistema \emph{Sharpener}, 
  em que novos Grupos de Exercícios Equivalentes são criados.}%
  \label{fig:add_track2}
  \end{figure}

  \begin{figure}[htpb]
  \centering
  \includegraphics[width=\linewidth]{images/mocks/trackAdd3.png}
  \caption{Página de trilhas da prova de conceito do sistema \emph{Sharpener}, 
	  em que Grupos de Exercícios Equivalentes são associados a passos de uma trilha, parte 1.}%
  \label{fig:add_track3}
  \end{figure}
  
  \begin{figure}[htpb]
  \centering
  \includegraphics[width=\linewidth]{images/mocks/trackAdd4.png}
  \caption{Página de trilhas da prova de conceito do sistema \emph{Sharpener}, 
	  em que Grupos de Exercícios Equivalentes são associados a passos de uma trilha, parte 2.}%
  \label{fig:add_track4}
  \end{figure}
  
  \begin{figure}[htpb]
  \centering
  \includegraphics[width=\linewidth]{images/mocks/exercicios.png}
  \caption{Página de exercícios da prova de conceito do sistema \emph{Sharpener}.}%
  \label{fig:exercicios}
  \end{figure}
  
  \begin{figure}[htpb]
    \centering
    \includegraphics[width=\linewidth]{images/mocks/turmaAddTrack.png}
    \caption{Página de turmas da prova de conceito do sistema \emph{Sharpener}, em 
	    que o professor inscreve suas turmas em trilhas.}%
    \label{fig:enroll_track1}
  \end{figure}
  
    \begin{figure}[htpb]
    \centering
    \includegraphics[width=\linewidth]{images/mocks/turmaAddTrackSearch.png}
    \caption{Página de turmas da prova de conceito do sistema \emph{Sharpener}, 
    que o professor inscreve suas turmas em trilhas, com um filtro  aplicado.}%
    \label{fig:enroll_track2}
  \end{figure}
\newpage


\section{A API (Application Programming Interface)}
Para suprir as necessidades da \emph{CLI} e do 
\emph{Frontend}, a \emph{API} fornece vários recursos diferentes. Alguns destes recursos não são protegidos, enquanto outros necessitam de identificação, que é provida pelo seu token pessoal. Os recursos disponíveis são
descritos a seguir, em que nomes entre o símbolo de maior e menor significam nomes variáveis 
da \emph{URI}.
\begin{description}
\item[\texttt{/api/classes}]  Rota protegida que, se o usuário for um professor, mostra todas as turmas que este criou. Caso o usuário seja aluno, mostra todas as classes em que faz parte. 
Aceita apenas o método \emph{GET}.
\item[\texttt{/api/classes/<name>}] Rota protegida que aceita apenas o método \emph{PUT}, emitido por 
um professor. Cria uma turma com o nome especificado. Turmas de um mesmo professor precisam 
ter nomes diferentes.
\item[\texttt{/api/classes/<name>/<track>}] Rota protegida que permite apenas o método \emph{PUT}, emitido por 
um professor. Associa uma turma, identificada pelo nome e criado por este mesmo professor, em uma determinada trilha.
\item[\texttt{/api/enrollments/<invite\_token>}] Rota protegida que aceita apenas o método \emph{POST}. Inscreve o Aluno em determinado turma, identificada por um código de convite.
\item[\texttt{/api/exercises}] Rota que aceita apenas o método \emph{GET}. Lista exercícios do banco 
de exercícios, por ordem ascendente por nível de dificuldade. A rota aceita paginação por meio 
de duas \emph{query strings}: ``\emph{page\_size}'' e ``\emph{page}''.
\item[\texttt{/api/exercises/<language>}] Rota que aceita apenas o método \emph{GET}. Lista exercícios do banco 
de exercícios, que sejam da linguagem especificada, por ordem ascendente por nível de dificuldade. A rota aceita paginação por meio 
de duas \emph{query strings}: ``\emph{page\_size}'' e ``\emph{page}''.
\item[\texttt{/api/exercises/<language>/<name>}] Rota protegida que aceita os método \emph{GET} e
\emph{PUT}. Ao receber um méotodo \emph{GET}, de um professor, mostra todos os detalhes relativos ao exercício, como a descrição do problema e \emph{URI}s para download dos artefatos associados. Se o ponto de acesso receber um \emph{PUT}, cria ou atualiza o exercício no banco de exercícios. Exercícios só podem ser atualizados se o criador do exercício foi o professor emissor da requisição.
\item[\texttt{/api/submissions/<submission\_token>}] Rota protegida que aceita  métodos \emph{POST} e \emph{GET}. O método \emph{GET} retorna todos os detalhes do exercício relativa a submissão referenciada 
por um \emph{token} de submissão e o método \emph{POST} registra uma tentativa de submissão.
\item[\texttt{/api/topics}] Rota que aceita apenas o método \emph{GET}. Mostra todos os tópicos já abordados por exercícios do banco de exercícios.
\item[\texttt{/api/topics/<language>}] Rota que aceita apenas o método \emph{GET}. Mostra todos os tópicos já abordados por exercícios do banco de exercícios, filtrados pela língua que foi 
provida na \emph{URI}.
\item[\texttt{/api/tracks/<track\_name>}] Rota protegida que aceita apenas o método \emph{PUT} de Professores. 
Cria uma nova trilha com o nome indicado na \emph{URI}, contanto que o professor não tenha criado outra com este mesmo nome.
\item[\texttt{/api/tracks/<track\_name>/classes/<class\_name>}] Rota protegida que aceita apenas o método \emph{PUT} de Professores. Matricula todos os alunos 
presentes na turma, que ainda não foram matriculados, na trilha que foi associada a uma turma específica. Ao invocar este ponto de acesso, o sistema cria 
, para cada aluno naquela turma e cada Grupo de Exercícios Equivalentes, uma submissão em estado pendente. Cada submissão é associada a um exercício aleatoriamente 
escolhido do Grupo de Exercícios Equivalentes. Se novos Alunos ingressarem na turma e uma nova requisição for disparada  ao ponto de acesso, submissões serão apenas 
criadas aos novos Alunos.
\item[\texttt{/api/healthcheck}] Rota que aceita apenas o método \emph{GET}. Informa se o serviço está funcionando corretamente e se a conexão com o banco de dados foi comprometida. Esta rota foi feita para que uma sonda externa faça esta consulta, e que caso haja alguma anormalidade, que um novo contêiner da aplicação seja iniciado e o tráfego direcionado a ele. 
\end{description}

Fora da \emph{API}, existem duas outras rotas, criadas especificamente para o \emph{Frontend}.
A primeira delas está exposta na raiz do servidor e serve o \emph{template} contendo o código da  \emph{single page application} do \emph{Frontend}. Uma outra rota, servida em ``\texttt{/authenticate}'', também se fez necessário para ser possível autenticação do usuário pelo \emph{Github}. A rota salva na sessão do usuário que está logado e suas informações, para que a maior parte dos acessos futuros não necessitem repetir os passos de autenticação federada.
% Pontos de acesso, etc. etc.

\subsection{Implementação} % DA API
A implementação do protótipo do servidor foi desenvolvida na linguagem \emph{Python} em conjunto 
com o \emph{microframework Web}, \emph{Flask}. 
% Discutir o protótipo do servidor: repassar as opções de cloud, banco, etc

\section{Resultados}
% FIM

\chapter{Conclusão}
\label{chapter:conclusao}
\section{Contribuições}
Apesar da prova de conceito desenvolvida ter se mostrado um primeiro passo na direção certa 
para contribuições na área de ensino de programação, o problema ainda se mostra complexo 
e longe de estar resolvido.

Espera-se, porém, que este projeto dê o ponto a pé inicial no desenvolvimento deste problema, 
com a prova de conceito já fornecendo bons alicerces para a construção de uma plataforma simples, 
robusta e atrativa para alunos.


% \chapter{Instalando o abnTeX2}
% \label{chapter:instalando-abntex}
% \input{tex/instalando-abntex}

% \chapter{Orientações Gerais}
% \label{chapter:orientacoes-gerais}
% \input{tex/orientacoes-gerais}

% \chapter{Configuração dos Elementos Pré-Textuais}
% \label{chapter:config-pre-textual}
% \input{tex/config-pre-textual}

% \chapter{Corpos Flutuantes}
% \label{chapter:corpos-flutuantes}
% \input{tex/corpos-flutuantes}

% \chapter{Listas}
% \label{chapter:listas}
% \input{tex/listas}

% \chapter{Ferramentas Úteis}
% \label{chapter:ferramentas-uteis}
% \input{tex/ferramentas-uteis}

% \chapter{Citações e Referências}
% \label{chapter:citacoes}
% \input{tex/citacoes}


% ---
% Finaliza a parte no bookmark do PDF, para que se inicie o bookmark na raiz
% ---
% \bookmarksetup{startatroot}% 
% ---

% ----------------------------------------------------------
% ELEMENTOS PÓS-TEXTUAIS
% ----------------------------------------------------------
% \postextual

% ----------------------------------------------------------
% Referências bibliográficas
% ----------------------------------------------------------
\bibliography{references}

% ---------------------------------------------------------------------
% GLOSSÁRIO
% ---------------------------------------------------------------------

% Arquivo que contém as definições que vão aparecer no glossário
% \input{tex/glossario}
% Comando para incluir todas as definições do arquivo glossario.tex
% \glsaddall
% Impressão do glossário
% \printglossaries

% ----------------------------------------------------------
% Apêndices
% ----------------------------------------------------------

% ---
% Inicia os apêndices
% ---
\begin{apendicesenv}

     \chapter{Detalhes de implementação}
     \label{chapter:implementacao}
     \section{Desenvolvendo a \emph{API} \emph{Web}}
Para o desenvolvimento da \emph{API}, escolheu-se a linguagem \emph{Python}, por sua clareza, 
e sua alta produtividade, que é crucial no processo de prototipação. Em conjunto com \emph{Python}, 
o \emph{micro framework web} \emph{Flask} também foi escolhido. Justifica-se essa escolha pela
inerente simplicidade, clareza e produtividade do \emph{framework}. 

Antes de podermos modelar recursos para a \emph{API},
estudaram-se quais dados eram necessários para os casos de uso propostos. Este estudo gerou um 
\emph{MER}, modelo entidade relacional, que relacionava todas as entidades do sistema. A partir 
do diagrama gerado, criou-se um conjunto de ``\emph{models}'' para representar estas entidades. A classe 
que possibilitou a criação destes \emph{models} provém da \emph{ORM} que abstrai bancos 
relacionais chamada \hyperref[link:sqlalchemy]{\emph{SQLAlchemy}}. \hyperref[link:sqlalchemy]{\emph{SQLAlchemy}}
é uma biblioteca estável com mais de treze anos de maturidade,
mas que continua sendo a escolha padrão dos desenvolvedores.

Optou-se por utilizar \hyperref[link:postgresql]{\emph{PostgreSQL}}.
A escolha foi feita por se tratar de um sistema 
gerenciador de banco de dados relacional de código aberto e por este ser referência em 
desempenho e funcionalidades.

No mapeamento do \emph{MER}, previamente esboçado, em classes utilizou-se a biblioteca 
de visualização \hyperref[link:eralchemy]{\emph{ERAlchemy}}. \hyperref[link:eralchemy]{\emph{ERAlchemy}} é capaz de gerar 
diagramas \emph{MER} de forma automática a partir das classes modelo ou de uma conexão com o banco de dados. 
Tal ferramenta se provou excepcionalmente útil pois permitiu o desenvolvimento incremental e iterativo das 
classes modelo e garantiu que o modelo conceitual fosse implementado sem divergências. A \fref{fig:mer} mostra 
o modelo entidade relacional do sistema, gerado a partir da ferramenta.

Para que nosso sistema fosse capaz de armazenar submissões de alunos de forma confiável,  
um serviço de intervalos de  armazenamento, traduzido do inglês \emph{Bucket Storage}, se 
fez necessário. Um intervalo de armazenamento nada mais é que uma abstração de provedores 
de computação em nuvem para oferecer armazenamento de objetos. Uma das grandes vantagens associadas 
a este tipo de serviço é o baixo custo por \emph{gigabyte}, além de sua alta escalabilidade, tanto 
 um aumento no volume de dados armazenados, quanto na velocidade recuperação destes.
O provedor de computação em nuvem escolhido foi a \hyperref[link:gcp]{\emph{Google Cloud Plataform}}.


A alternativa a adotar um serviço de armazenamento por intervalos seria armazenar arquivos 
diretamente no banco de dados, o que traria um impacto na performance do mesmo, já que \emph{SGBD}s não são otimizados para
este tipo de dado. 

Modelaram-se recursos seguindo o estilo arquitetural \emph{REST}. Coleções e documentos
de exercícios, turmas, tópicos, entre outros recursos foram implementados, e um 
recurso do arquétipo \emph{controller} foi necessário para prover autenticação a interface por 
meio do fluxo de concessão do código de autorização. Também disponibilizou-se uma rota para \emph{health check}, 
que permite que uma sonda externa consulte se a aplicação continua funcionando adequadamente. Caso esta não esteja, 
envia-se um sinal para que um novo contêiner da aplicação seja iniciado e o tráfego é redirecionado a ela. 

Para que não seja necessário que professores criem um banco de exercícios do zero, um ``conector'' foi 
implementado para a plataforma \emph{exercism}, em que seus exercícios são extraídos e guardados no banco 
de dados. A implementação aborda apenas duas linguagens \emph{Python} e \emph{Rust}, mas 
pode ser facilmente estendido para outras linguagens, dado que se informe a estrutura de arquivos 
que os exercícios daquela linguagem são armazenados.

Também foi configurado uma plataforma de \emph{CI/CD} para o projeto. A toda nova versão enviada 
ao repositório remoto do sistema de controle de versões, uma tarefa rodava a ferramenta \emph{autopep8}
que checava se o código \emph{commitado} era sintaticamente válido e não seguia más práticas. 
Caso o código fosse reprovado na tarefa anterior, este não poderia ser aprovado e mesclado 
na \emph{branch} principal de desenvolvimento. Se a contribuição passar na verificação e 
depois de revisão for aceita, uma tarefa para lançamento da nova versão é disparada automaticamente, 
       que coloca uma nova versão no ar. A plataforma de \emph{CI/CD} empregada foi o 
       \hyperref[link:actions]{\emph{Github Actions}} e 
       as novas versões eram colocadas no ar através da \emph{Platform as a Service} do \emph{Google}, 
       \hyperref[link:appengine]{\emph{AppEngine}}.  \\ 
       O \Aref{capitulo:cicd} conceitua com mais detalhes o que é integração e  entrega contínua .



\section{Desenvolvendo a CLI}
O desenvolvimento da \emph{CLI} foi feito na linguagem \emph{Rust}, utilizando 
a biblioteca \emph{StructOpt}.  \emph{StructOpt} é uma biblioteca que permite construção 
de interfaces por linhas de comando a partir da anotação de macros em \emph{structs} ou 
\emph{enums}. Com uma simples anotação ganha-se um \emph{parser} dos comandos e mensagens 
que instruem o usuário como utilizar sua \emph{CLI}. Também utilizou-se a biblioteca \emph{reqwest} que fornece um cliente \emph{HTTP} que abstraí
a construção de requisições e facilita o consumo de respostas.

\section{Desenvolvendo a interface}
A interface foi desenvolvida utilizando \emph{Javascript}, \emph{CSS} e \emph{HTML}. Como 
desejávamos uma plataforma que fosse bastante interativa, uma \emph{single page application} 
foi implementada, utilizando a tecnologia \hyperref[link:react]{\emph{React}}.
Seguiu-se o \emph{design system} chamado 
\emph{Material Design} por ser bastante intuitivo para novos usuários. A biblioteca \emph{Material-UI} 
forneceu vários componentes que serviram de base para os componentes customizados. A comunicação com a 
\emph{API} foi feita utilizando o cliente \emph{HTTP} \hyperref[link:axios]{\emph{axios}} e 
os dados são persistidos numa \emph{store} local, utilizando o gerenciador de estados \emph{Redux}.

\begin{sidewaysfigure}[htpb]
    \centering
    \includegraphics[width=\linewidth]{images/db_schema}
    \caption{Modelo entidade relacional gerado a partir da ferramenta \hyperref[link:eralchemy]{\emph{ERAlchemy}}.}\label{fig:mer}
\end{sidewaysfigure}
    \chapter{Integração e entrega contínua}
     \label{chapter:cicd}
     \label{capitulo:cicd}
  Um dos grandes desafios no desenvolvimento de \emph{software} é criar um processo repetível e 
  confiável para entregas de aplicações. Frequentemente, lançamentos de \emph{software} são
  tratados por seus desenvolvedores como momentos estressantes. Associa-se este \emph{stress} a
  muitos processos manuais, infrequentes e propensos a erros, conduzidos em curtos prazos 
  de tempo.
  
  Pode-se, no entanto, se um processo rigoroso for seguido,  
  tornar esta tarefa fácil, tão fácil quanto o apertar de um botão.
  Para atingirmos este nível de maturidade no desenvolvimento de 
  \emph{software}, \citeonline{humble2010continuous} propõem que
  devemos seguir à risca dois princípios: automatização de todas etapas que sejam 
  automatizáveis e manter todos os artefatos necessários para um lançamento em um sistema de 
  controle de versões. 

  Idealmente, uma entrega de \emph{software} é composta por três atividades: 
  provisionar e gerenciar o ambiente em que a aplicação irá rodar (configuração de \emph{hardware},
  \emph{software}, infraestrutura e serviços externos), instalar a versão correta da aplicação 
  neste ambiente e configurar a aplicação, incluindo qualquer estado ou dado que ela possa requerer.

  De fato, até recentemente, vários destes passos para a entrega de \emph{software} pareciam 
  impossíveis ou ao menos complexas de aderirem os princípios citados por 
  \citeonline{humble2010continuous}. Como poderíamos, por exemplo, 
  versionar \emph{hardware}? Com o advento de virtualização eficiente e barata, até esta 
  tarefa que, \emph{a priori}, parecia impossível, virou algo corriqueiro e ubíquo. 
  Com advento e adoção da computação em nuvem, todos os passos necessários para automatização 
  de testes, compilação ou de artefatos e lançamento do \emph{software} estão 
  acessíveis a qualquer desenvolvedor.

  Define-se integração contínua como o processo no desenvolvimento de \emph{software} em que 
  membros de um time integram seu trabalho de forma frequente. Cada integração desencadeia etapas 
  como verificação de estilo de código, testes automatizados e construções de compilados ou 
  outros artefatos, com o objetivo de encontrar problemas nas integrações o mais rápido possível.
  Já entrega contínua leva a prática de integração contínua para outro patamar e automatiza 
  o lançamento de versões do \emph{software} dado que as etapas anteriores foram concluídas com 
  sucesso e o código foi revisado \cite{fowler2006continuous}.


\end{apendicesenv}


% ----------------------------------------------------------
% Anexos
% ----------------------------------------------------------

% ---
% Inicia os anexos
% ---
\begin{anexosenv}
    \chapter{Links relacionados} 
    \label{chapter:paginas-interessantes}
    \begin{description}
\label{link:the_huxley}
  \item[\url{https://www.thehuxley.com}:]  Página da plataforma \emph{The Huxley}, 
    que utiliza um sistema de juiz \emph{online} para auxiliar professores em aulas de programação.

\label{link:code_bench}
  \item[\url{http://codebench.icomp.ufam.edu.br}:]  Página da plataforma \emph{CodeBench}, 
    que utiliza um sistema de juiz \emph{online}, de forma gamificada, para auxiliar professores em aulas de programação.

\label{link:uva_judge}
  \item[\url{https://uva.onlinejudge.org}:] Página da plataforma \emph{UVa Online Judge}, um sistema
    de juiz \emph{online} desenvolvido pela \emph{University of Valladolid}.

\label{link:feeper}
  \item[\url{http://feeper.unisinos.br}:] Página da plataforma \emph{UVa Online Judge}, um sistema
    de juiz \emph{online} desenvolvido pela \emph{Universidade do Vale do Rio dos Sinos}.

\label{link:uri_judge}
  \item[\url{https://www.urionlinejudge.com.br}:] Página da plataforma \emph{URI Online Judge}, um sistema
    de juiz \emph{online} desenvolvido pela \emph{Universidade Regional Integrada}.

\label{link:boca}
  \item[\url{https://www.ime.usp.br/~cassio/boca}:] Página da plataforma \emph{BOCA}, um sistema
    de juiz \emph{online} desenvolvido pelo \emph{Instituto de Matemática e Estatística} para apoio 
    em competições.

\label{link:we_run_codes}
  \item[\url{https://we.run.codes}:] Página da plataforma \emph{RunCodes}, um sistema
    de juiz \emph{online} desenvolvido pelo \emph{Instituto de Matemática e Estatística}, 
    para auxiliar professores em aulas de programação.

\label{link:sphere_judge}
  \item[\url{https://www.spoj.com}:] Página da plataforma \emph{Sphere Online Judge}, um sistema
    de juiz \emph{online} desenvolvido pela \emph{Sphere Research Labs}.

\label{link:hacker_rank}
  \item[\url{https://www.hackerrank.com}:]  Página da plataforma \emph{HackerRank}, 
    que utiliza um sistema de juiz \emph{online} para auxiliar na contratação de funcionários 
    para empresas.

\label{link:code_chef}
  \item[\url{https://www.codechef.com}:]  Página da plataforma \emph{CodeChef}, 
    que utiliza um sistema de juiz \emph{online}, desenvolvida pela empresa \emph{Directi},
    para competições em programação.

\label{link:interview_bit}
  \item[\url{https://www.interviewbit.com}:]  Página da plataforma \emph{HackerRank}, 
    que utiliza um sistema de juiz \emph{online} para auxílio na preparação de entrevistas
    para empresas.

\label{link:kattis}
  \item[\url{https://open.kattis.com}:] Página da plataforma \emph{Kattis}, um sistema
    de juiz \emph{online} desenvolvido para promover competições.

    \label{link:leet_code}
  \item[\url{https://leetcode.com}:]  Página da plataforma \emph{LeetCode}, 
    que utiliza um sistema de juiz \emph{online} para auxílio na preparação de entrevistas
    para empresas.

    \label{link:codin_game}
  \item[\url{https://www.codingame.com}:]  Página da plataforma \emph{CodinGame}, 
    que utiliza um sistema de juiz \emph{online}, de forma completamente gamificada,
    para ensino e auxiliar empresas na contratação de funcionários.

    \label{link:code_signal}
  \item[\url{https://codesignal.com}:]  Página da plataforma \emph{CodeSignal}, 
    que utiliza um sistema de juiz \emph{online} para auxiliar empresas 
    na contratação de funcionários.

    \label{link:code_wars}
  \item[\url{https://www.codewars.com}:]  Página da plataforma \emph{CodeWars}, 
    que utiliza um sistema de juiz \emph{online} para auxiliar professores em aulas 
    de programaçãoe e empresas na contratação de funcionários.

    \label{link:exercism}
  \item[\url{https://exercism.io}:]  Página da plataforma \emph{Exercism}, 
    que utiliza um sistema de juiz \emph{online}, para ensino. As submissões
    contam com feedback de instrutores voluntários.

    \label{link:so_survey}
  \item[\url{https://insights.stackoverflow.com/survey/2019}:] Resultado de 2019 da pesquisa 
    anual de desenvolvedores do site \emph{StackOverflow}. É a maior e 
    mais abrangente pesquisa que envolve desenvolvedores.

    \label{link:oauth}
    \item[\url{https://docs.oracle.com/cd/E39820_01/doc.11121/gateway_docs/content/oauth_flows.html}:]
Documentação do \emph{software API Gateway}, desenvolvido pela \emph{Oracle Corporation}.

\label{link:sqlalchemy}
\item[\url{https://www.sqlalchemy.org}:] \emph{SQLAlchemy} trás um 
  conjunto de ferramentas e uma \emph{ORM} para utilização de bancos de dados relacionais 
  na linguagem \emph{Python}.

  \label{link:postgresql}
\item[\url{https://www.postgresql.org}:] Página do site do banco relacional e de código aberto 
  \emph{PostgreSQL}.
  


\label{link:eralchemy}
\item[\url{https://pypi.org/project/ERAlchemy}:] Biblioteca para \emph{Python}, chamada \emph{ERAlchemy}, 
  capaz de gerar diagramas entidade relacional através de \emph{models} do \emph{SQLAlchemy} ou de uma conexão 
  com banco de dados.

  \label{link:gcp}
\item[\url{https://cloud.google.com}:] Página do fornecedor de computação em nuvem, \emph{Google 
    Cloud Platform}.

  \label{link:actions}
\item[\url{https://github.com/features/actions}:] Página para a plataforma de \emph{Continuous Integration}
    e \emph{Continuous Delivery} do site \emph{Github}.

    \label{link:appengine}
\item[\url{https://cloud.google.com/appengine}:] Página para o \emph{Platform as a Service} da 
  \emph{Google Cloud Plataform}, chamado \emph{AppEngine}.

  \label{link:react}
\item[\url{https://pt-br.reactjs.org}:] \emph{Framework} javascript para a construção 
  de interfaces em \emph{Single Page Applications}.

  \label{link:mui}
\item[\url{https://material-ui.com}:] Biblioteca que implementa o \emph{design system} da 
  \emph{Google}, \emph{Material Design}, em \emph{React}.

  \label{link:axios}
\item[\url{https://github.com/axios/axios}:] Página do mais popular cliente \emph{HTTP} 
  \emph{javascript} da atualidade.

  \label{link:redux}
\item[\url{https://redux.js.org}:] Página para a biblioteca \emph{Redux}, que implementa 
  parte da arquitetura \emph{FLUX}. \emph{Redux} é um gerenciador de estados previsível 
  e centralizado para aplicações \emph{Web}.

  \label{link:struct_opt}
\item[\url{https://github.com/TeXitoi/structopt}:] Biblioteca em \emph{Rust} que gera 
  interfaces por linha de comando a partir de \emph{structs} anotadas com macros. 
  
  \label{link:api}
\item[\url{https://github.com/abbudao/sharpener}:] Página do repositório da \emph{API Web} do sistema sharpener, 
desenvolvido com \emph{Python}, \emph{Flask} e \emph{SQLAlchemy}.

  \label{link:cli}
\item[\url{https://github.com/abbudao/sharpener-cli}:] Página do repositório da \emph{CLI} do sistema sharpener, 
desenvolvido em \emph{Rust} com a biblioteca \emph{StructOpt}.

  \label{link:frontend}
\item[\url{https://github.com/abbudao/sharpener-frontend}:] Página do repositório do \emph{Frontend} do sistema sharpener, 
desenvolvido em \emph{Javascript}, utilizando \emph{React}.
\end{description}

\end{anexosenv}
% ---

\end{document}
