\label{chapter:conclusao}
\section{Contribuições}
O presente trabalho pode ser entendido como um exercício de \emph{design} de software educativo para o ensino de programação, com destaque para as funcionalidades de apoio ao aluno, tais como: 
\begin{itemize}
    \item acesso a dicas por exercício, 
    \item acesso à resposta com possibilidade de resolver um exercício alternativo, e 
    \item exercícios formulados de forma incremental.
\end{itemize}
O acompanhamento de alunos com base nas formas de apoio acionadas por eles também é um diferencial.

Como prova de conceito, foi desenvolvido um protótipo funcional, para o qual foram parcialmente projetados e implementados: 
\begin{itemize}
    \item acesso através de uma API web, 
    \item um modelo conceitual para os dados, 
    \item interface de linha de comando e 
    \item interface Web.
\end{itemize}
O protótipo funciona em nuvem, integra tecnologias de autenticação federada,
e pode ser um ponto de partida para a continuação do desenvolvimento da
ferramenta \emph{Sharpener}.
\section{Trabalhos futuros}
\begin{itemize}
    \item Finalização das funcionalidades básicas do protótipo, especialmente a interface administrativa (cliente Web), com a finalidade de viabilizar a aplicação em situações reais.   
    \item É importante buscar validar as ideias a partir da aplicação do sistema em disciplinas de programação.
    Em particular, deve-se observar como o aluno se apropria das formas diferenciadas de apoio, e qual o impacto disso na evolução do aluno ao longo da disciplina.
    \item A ideia de geração parcialmente aleatória de exercícios tem afinidade com a ideia de Grupo de Exercícios Equivalentes/Alternativos. A proposta é criar \emph{templates} de enunciados com pontos de inserção a serem preenchidos aleatoriamente. Os \emph{templates} se estenderiam também aos casos de teste, e às soluções armazenadas.
    \item Uma ideia discutida durante a concepção, era uma painel de anotação das respostas, que permitiria
    ao professor montar uma base de dados de erros e soluções mais recorrentes, e classificá-los mediante alguma taxonomia, podendo também fazer buscas. Essa base de dados serviria como plataforma para o desenvolvimento de exercícios. 
    \item O sistema deveria aceitar a importação de ontologias educacionais para objetivos pedagógicos, avaliação e outras formas de integração com sistemas educacionais.
\end{itemize}

