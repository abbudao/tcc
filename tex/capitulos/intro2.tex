\section{Motivação e Contextualização}
Nas últimas décadas, o interesse por programação cresceu de forma extraordinária 
e cursos introdutórios mostram-se cada vez mais populares. No entanto, 
cursos de programação ainda são considerados difíceis e, com frequência,
possuem taxas de desistência elevada. Iniciantes sofrem de uma 
vasta gama de défices e dificuldades, desde a compreensão de constructos 
básicos de uma linguagem de programação à elaboração de planos
para resolver um dado problema \ \cite{robins2003learning}.

Estudos concluem, de forma semelhante, que alunos, 
mesmo tendo aprendido a sintaxe e semântica de uma linguagem 
de programação, tendem a falhar em compor os blocos fundamentais 
estudados em programas válidos. Parte deste problema é abordada por 
\citeonline{davies1993models}, que faz uma distinção de duas partes 
essenciais no aprendizado em programação: 
conhecimento e estratégia. Denota-se conhecimento em programação como a parte declarativa 
do conhecimento, por exemplo, saber que existe um constructo chamado \emph{for} na linguagem 
e seu propósito. Já estratégia em programação seria empregar de forma adequada o conhecimento 
em face de um problema, como o uso pertinente de um \emph{for} em um programa.
\citeonline{davies1993models} defende que grande parte das abordagens em ensino 
tendem a investir excessivamente na parcela de conhecimento em programação, muitas 
vezes deixando de lado o aspecto de estratégia.

Em vista do imenso desafio que é o ensino de programação a iniciantes, diversas ferramentas 
foram desenvolvidas para dar apoio pedagógico ao professor nesta tarefa.  
Estas ferramentas, em grande parte, são sistemas de juízes \emph{online}, 
que consistem em sistemas capazes 
de receber códigos desenvolvidos pelos usuários como resposta a um determinado problema, mas 
com a capacidade de dar \emph{feedback} imediato ao usuário, informando 
se o código está correto ou não, através de análise sintática e testes automatizados.

Já existe uma grande variedade em juízes \emph{online} disponíveis na literatura científica,
tais como: 
\hyperref[link:the_huxley]{\emph{The Huxley}} [\citeonline{de2013ferramenta}], 
	\hyperref[link:code_bench]{\emph{CodeBench}} [\citeonline{galvao2016juiz}],  
	\hyperref[link:uva_judge]{\emph{UVa Online Judge}} [\citeonline{revilla2008competitive}], 
	\hyperref[link:feeper]{\emph{feeper}} [\citeonline{alves2014ambiente}], 
	\hyperref[link:uri_judge]{\emph{URI Online Judge}} [\citeonline{bez2014uri}], 
	\hyperref[link:boca]{\emph{BOCA}} [\citeonline{de2004boca}],
	\hyperref[link:we_run_codes]{\emph{RunCodes}},
	\hyperref[link:sphere_judge]{\emph{Sphere Online Judge}},
	\hyperref[link:hacker_rank]{\emph{HackerRank}}, 
	\hyperref[link:code_chef]{\emph{CodeChef}}, 
	\hyperref[link:interview_bit]{\emph{InterviewBit}}, 
	\hyperref[link:kattis]{\emph{Kattis}}, 
	\hyperref[link:leet_code]{\emph{LeetCode}}, 
	\hyperref[link:codin_game]{\emph{CodinGame}}, 
	\hyperref[link:code_signal]{\emph{CodeSignal}}, 
	\hyperref[link:code_wars]{\emph{CodeWars}}, 
	\hyperref[link:exercism]{\emph{Exercism}}, 
	entre tantos outros.

Boa parte dessas ferramentas consiste em serviço gratuito oferecido ao aprendiz,
e com poucos recursos de acompanhamento por parte dos professores.
Outra parte, sim, é direcionada para situações acadêmicas, mas com 
\emph{workflows} de submissão e correção de programas rígidos;
qualquer intenção de personalizar uma típica ferramenta desse gênero
implica alterar o código-fonte.

O sistema que é tema do presente trabalho, o \emph{Sharpener}, surgiu do
anseio de se fazer pequenas adaptações a ferramentas de apoio a disciplinas
de programação.
Um problema observado é que alguns alunos não sabem como dar um próximo passo
na resolução de um exercício;
muitas vezes, eles precisam de um empurrão (como uma dica) para prosseguir.
Em aulas com muitos alunos, o professor nem sempre é capaz de proporcionar
esse tipo de acompanhamento e atenção, devido às limitações evidentes.
A questão passa a ser, então, se uma ferramenta poderia prover ao aluno
algum tipo de apoio, de forma que ele, seguindo a ferramenta, pudesse
explorar possibilidades e alternativas que o impulsionassem na continuação
do exercício; o aluno teria, com isso, uma relativa autonomia no aprendizado.

A solução mais imediata é capacitar a ferramenta a entregar dicas conforme
o aluno as solicita. Para isso, o professor precisaria ter alimentado o sistema
com dicas no mesmo momento em que teria carregado o enunciado do exercício.
Uma ideia similar é entregar ao aluno, de uma vez, a resposta completa do problema.
Muitos alunos têm uma ansiedade por ver respostas. 
A experiência de ver a resposta de um problema não resolvido pode ser um momento
de intensa aprendizagem.
Em vez de coibir isso, a ferramenta pode estimular, compensando com a 
apresentação, em seguida, de um exercício equivalente e alternativo.
Uma terceira proposta, ainda no espírito de fornecer algum tipo de dica,
é conduzir o aluno numa solução incremental do problemas, propondo pequenos
passos (incrementos) rumo à solução.
Com \emph{workflows} como esses, além do efeito pretendido de dar alguma 
autonomia ao aluno, há um efeito colateral que é a possibilidade de o professor
ter acesso ao histórico de consultas do aluno, podendo fazer uma avaliação
mais rica do seu percurso de aprendizagem.

Em vez de adaptar ferramentas existentes cujo código fosse aberto,
optou-se por criar um protótipo de uma ferramenta nova como prova de conceito.
O foco desse desenvolvimento passou a ser as componentes arquiteturais 
principais do sistema, como o servidor e sua API de acesso, e
uma ferramenta cliente de linha de comando.
Embora a contribuição pretendida aqui resida no domínio do
ensino de programação e suas ferramentas de apoio,
levando em conta que se trata de um Trabalho de Conclusão de Curso,
é inevitável tentar apresentar neste documento
alguns aspectos predominantes no desenvolvimento do protótipo, 
tais como: \emph{design} de APIs Web, estilo arquitetural \emph{REST}, computação em nuvem, autenticação federada,
e as principais decisões de projeto.
O escopo proposto inicialmente revelou-se excessivo, considerando-se o prazo de
um semestre letivo, e este trabalho passou a ser apenas a etapa inicial do
desenvolvimento do \emph{Sharpener}.

\section{Objetivos}
Pretende-se neste trabalho desenvolver um protótipo, como prova de conceito, de um sistema para apoio 
ao ensino de programação em salas de aula, que forneça apoio a estudantes estagnados no
desenvolvimento de exercícios, apoio esse que se manifesta
na forma de dicas, respostas, exercícios alternativos e
passos incrementais.

Em termos de software, pretende-se desenvolver minimamente uma arquitetura baseada em Web,
com um servidor acessível por \emph{API REST}, com suporte a autenticação federada, 
compatível com computação em nuvem, um cliente de linha de comando para funcionalidades
diretamente ligadas aos exercícios, e um cliente web para
funcionalidades administrativas básicas.

\section{Organização}
O cerne do trabalho é apresentado no \Cref{chapter:desenvolvimento}. Ali se encontram: a descrição
dos \emph{workflows} do principais atores (Aluno e Professor), seja na interface de linha de comando ou
na interface Web; e a descrição da API Web.

Dado o caráter de desenvolvimento de software do presente trabalho, diversos tópicos
foram mobilizados para a sua execução, entre metodologias, técnicas e ferramentas.
No \Aref{chapter:implementacao}, traça-se um panorama de todo o processo.
O \Cref{chapter:metodos} contém uma apresentação de alguns tópicos selecionados,
assim como o \Aref{chapter:cicd}.

Por fim, no \Cref{chapter:conclusao} estão as conclusões e contribuições do projeto, 
seguidas dos trabalhos futuros.