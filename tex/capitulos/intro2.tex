\section{Motivação e Contextualização}
Nas últimas décadas, o interesse por programação cresceu de forma extraordinária 
e cursos introdutórios mostram-se cada vez mais populares. No entanto, 
cursos de programação ainda são considerados difíceis e, com frequência,
possuem taxas de desistência elevada. Iniciantes sofrem de uma 
vasta gama de défices e dificuldades, desde a compreensão de constructos 
básicos de uma linguagem de programação à elaboração de planos
para resolver um dado problema \ \cite{robins2003learning}.

Estudos concluem, de forma semelhante, que alunos, 
mesmo tendo aprendido a sintaxe e semântica de uma linguagem 
de programação, tendem a falhar em compor os blocos fundamentais 
estudados em programas válidos. Parte deste problema é abordada por 
\citeonline{davies1993models}, que faz uma distinção de duas partes 
essenciais no aprendizado em programação: 
conhecimento e estratégia. Denota-se conhecimento em programação como a parte declarativa 
do conhecimento, por exemplo, saber que existe um constructo chamado \emph{for} na linguagem 
e seu propósito. Já estratégia em programação seria empregar de forma adequada o conhecimento 
em face de um problema, como o uso pertinente de um \emph{for} em um programa.
\citeonline{davies1993models} defende que grande parte das abordagens em ensino 
tendem a investir excessivamente na parcela de conhecimento em programação, muitas 
vezes deixando de lado o aspecto de estratégia.

Em vista do imenso desafio que é o ensino de programação a iniciantes, diversas ferramentas 
foram desenvolvidas para dar apoio pedagógico ao professor nesta tarefa.  
Estas ferramentas, em grande parte, são sistemas de juízes \emph{online}, 
que consistem em sistemas capazes 
de receber códigos desenvolvidos pelos usuários como resposta a um determinado problema, mas 
com a capacidade de dar \emph{feedback} imediato ao usuário, informando 
se o código está correto ou não, através de análise sintática e testes automatizados.

Já existe uma grande variedade em juízes \emph{online} disponíveis na literatura científica,
tais como: 
\hyperref[link:the_huxley]{\emph{The Huxley}} [\citeonline{de2013ferramenta}], 
	\hyperref[link:code_bench]{\emph{CodeBench}} [\citeonline{galvao2016juiz}],  
	\hyperref[link:uva_judge]{\emph{UVa Online Judge}} [\citeonline{revilla2008competitive}], 
	\hyperref[link:feeper]{\emph{feeper}} [\citeonline{alves2014ambiente}], 
	\hyperref[link:uri_judge]{\emph{URI Online Judge}} [\citeonline{bez2014uri}], 
	\hyperref[link:boca]{\emph{BOCA}} [\citeonline{de2004boca}],
	\hyperref[link:we_run_codes]{\emph{RunCodes}},
	\hyperref[link:sphere_judge]{\emph{Sphere Online Judge}},
	\hyperref[link:hacker_rank]{\emph{HackerRank}}, 
	\hyperref[link:code_chef]{\emph{CodeChef}}, 
	\hyperref[link:interview_bit]{\emph{InterviewBit}}, 
	\hyperref[link:kattis]{\emph{Kattis}}, 
	\hyperref[link:leet_code]{\emph{LeetCode}}, 
	\hyperref[link:codin_game]{\emph{CodinGame}}, 
	\hyperref[link:code_signal]{\emph{CodeSignal}}, 
	\hyperref[link:code_wars]{\emph{CodeWars}}, 
	\hyperref[link:exercism]{\emph{Exercism}}, 
	entre tantos outros.

Boa parte dessas ferramentas consiste em serviço gratuito oferecido ao aprendiz,
e com poucos recursos de acompanhamento por parte dos professores.
Outra parte, sim, é direcionada para situações acadêmicas, mas com 
\emph{workflows} de submissão e correção de programas rígidos;
qualquer intenção de personalizar uma típica ferramenta desse gênero
implica alterar o código-fonte.

O sistema que é tema do presente trabalho, o \emph{Sharpener}, surgiu do
anseio de se fazer pequenas adaptações a ferramentas de apoio a disciplinas
de programação.
Um problema observado é que alguns alunos não sabem como dar um próximo passo
na resolução de um exercício;
muitas vezes, eles precisam de um empurrão (como uma dica) para prosseguir.
Em aulas com muitos alunos, o professor nem sempre é capaz de proporcionar
esse tipo de acompanhamento e atenção, devidos às limitações evidentes.
A questão passa a ser, então, se uma ferramenta poderia prover ao aluno
algum tipo de apoio, de forma que o aluno, seguindo a ferramenta, pudesse
explorar possibilidades e alternativas que o impulsionassem na continuação
do exercício; o aluno teria, com isso, uma relativa autonomia no aprendizado.

A solução mais imediata é capacitar a ferramenta e entregar dicas conforme
o aluno as solicita. Para isso, o professor precisaria ter alimentado o sistema
com dicas no mesmo momento em que teria carregado o enunciado do exercício.
Uma ideia similar é entregar ao aluno, de uma vez, a resposta completa do problema.
Muitos alunos têm uma ansiedade por ver respostas. 
Um aluno que passou


poucas 
têm como foco ser uma ferramenta de auxílio didático ao professor em sala de aula e 
praticamente nenhuma é de código aberto. Assim sendo, existe uma lacuna que pode
ser preenchida por um sistema que tenha estes dois focos e dê liberdade ao professor 
a implementar técnicas de ensino que o convenha.


\section{Objetivos}
Pretende-se neste trabalho desenvolver uma prova de conceito de um sistema para apoio 
ao ensino de programação em salas de aula, que seja de código aberto e que forneça 
ferramentas necessárias para o professor conduzir e monitorar exercícios práticos dados 
em sala de aula.

\section{Organização}
No Capítulo 2 pretende-se fazer uma revisão bibliográfica das técnicas, métodos e tecnologias
empregadas na implementação 
da prova de conceito do sistema \emph{Sharpener}. A seguir, no Capítulo 3, descreve-se a especificação 
funcional do sistema tais como seus detalhes de implementação. Finalmente no Capítulo 4 apresentam-se 
os resultados e possíveis trabalhos futuros.
