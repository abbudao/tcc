\label{chapter:metodos}
\section{Considerações Iniciais}
O desenvolvimento de uma aplicação Web com suporte a computação em nuvem
requer a articulação de muitas tecnologias, ferramentas, metodologias.
Neste capítulo, alguns temas são repassados e discutidos, como 
o \emph{design} de APIs Web, que é central no trabalho.
Além disso, o \emph{framework Flask}, a linguagem \emph{Rust} e 
a autenticação federada são temas relativamente recentes, e, assim,
vêm em seguida.


\section{Design de \emph{API}s \emph{Web}}
Separação de conceitos, traduzido do termo \emph{Separation of Concerns}, é uma temática 
chave na arquitetura cliente-servidor da internet. Parte do porquê a internet funciona tão bem
foi a preocupação, desde sua concepção, de uma interface uniforme que, desde que respeitada, 
 daria liberdade aos desenvolvedores de implementar, em qualquer linguagem ou tecnologia,
 um de seus componentes. 

Apesar do sucesso na separação de conceitos entre as responsabilidades que o navegador 
e o servidor empregam, a mesma preocupação não foi replicada entre interfaces e 
funcionalidades que o servidor expõe. 
Com frequência existe um alto acoplamento entre a interface, \emph{Frontend},
e o servidor, \emph{Backend}, o que resulta em uma interface não uniforme para interação 
dos recursos que deveriam estar expostos. Define-se aqui um recurso como qualquer conceito 
na internet que pode ser referenciado por um identificador único e manipulado por uma interface uniforme 
\ \cite{masse2011rest}.

A solução, que leva ao desacoplamento, é a interação com estas
funcionalidades través de uma \emph{API}, \emph{Application Programming Interface},
exposta pelo \emph{backend}, que provê formas padronizadas 
de acesso. Aqui não somente ganhamos portabilidade, já que possibilitamos que não só 
um tipo de cliente, navegadores, saibam como acessar nossos recursos, mas ganhamos espaço 
para criar uma camada de abstração do serviço \emph{Web}, modelando-o em recursos.
Estes recursos não serão projetados como uma cópia da organização de dados ou funcionalidades 
presentes no \emph{backend} mas em representações que sejam de fácil consumo e de entendimento 
ubíquo pelo lado do cliente.

O desafio de criar bons serviços na internet pode ser facilitado se empregarmos estilos 
arquiteturais já existentes. Aqui definimos estilos arquiteturais como 
um conjunto coordenado de restrições arquiteturais. \\
Um destes estilos arquiteturas que tem ganhado cada vez 
mais tração se chama \emph{REST}, \emph{Representational State Transfer}, ou em português,
\emph{Transferência Representacional de Estado}. Cunhado por \citeonline{fielding2000architectural}, 
o termo evoca como um sistema na \emph{Web} deveria se comportar, uma máquina de estados 
virtuais em que o usuário progride através da seleção de identificadores únicos, que identificam 
recursos, e verbos \emph{HTTP}
que operam sobre estes recursos.

As restrições na arquitetura impostas pelo estilo \emph{REST} são agrupadas em seis categorias:
\begin{description}
\item[Cliente-servidor:] A separação dos papéis do cliente e servidor deve ser clara para 
  que estes possam ser projetados e implementados de forma independente. A interação 
  entre eles só acontece na forma de requisições, que são iniciadas pelo cliente. 
  Servidores devem mandar respostas apenas como reações de requisições dos clientes.
\item[Interface uniforme:] interfaces \emph{REST} possuem quatro  restrições:
identificação de recursos, manipulação de recursos através de representações, 
mensagens auto-descritivas e hipermídia como motor de estado da aplicação, \emph{HATEOAS}.\\
A primeira dita que cada recurso precisa ser endereçável 
por um identificador único, \emph{URI}, \emph{Unique Resource Identifier}.\\
Este 
recurso pode ser representado de diversas formas, seja em formato \emph{HTML}, que é mais adequado para 
um navegador,
ou formato \emph{JSON}, que é mais apropriado para consumo de outro programa. Percebe-se aqui 
que a representação é apenas uma forma de interagir com o 
recurso,  não o próprio recurso. Isto é o que dita a segunda regra. \\
No consumo de uma \emph{API}, o cliente especifica um recurso e seu estado 
desejado, enquanto o servidor deve responder com o recurso e seu estado real. 
Esta troca de mensagens  deve ser feita utilizando \emph{headers} e 
códigos de estado \emph{HTTP} que descrevam o estado 
do recurso e metadados correspondentes, de forma que as mensagens sejam auto-descritivas.\\
A última restrição, \emph{HATEOAS}, ajuda a aumentar a visibilidade e descoberta de recursos 
relacionados na \emph{API}, que ajudam o cliente a navegar dinamicamente nos recursos expostos.
Assim, quando referenciam-se outros recursos, também são fornecidos seus identificadores únicos 
e verbos aceitos para aquela rota.
\item[Sistema em camadas:] Restringe-se o comportamento dos componentes em camadas, em que 
  cada camada só pode interagir com subjacentes. 
  Entre as chamadas do cliente que requisita uma representação de um estado de um recurso, e 
  o servidor que processa a requisição, pode haver vários servidores entre eles. Estes 
  servidores podem prover um camada de segurança, \emph{cache}, balanceamento de carga ou 
  outras funcionalidades. Estas camadas não devem afetar a requisição ou resposta e nem o cliente 
  nem servidor precisam estar cientes se elas existem ou não.
\item[Protocolo sem estado:] O servidor não deve lembrar absolutamente nada do usuário que 
  utiliza sua \emph{API}. Isto implica que toda requisição individual precisa conter toda 
  informação necessária para que o servidor processe e retorne uma resposta. Esta restrição 
  visa aumentar a escalablidade, visibilidade e confiabilidade do sistema. Escalabilidade pois 
  permite que haja \emph{caching} de respostas e que o servidor possa desalocar recursos 
  físicos entre requisições, visibilidade pois sistemas que monitoram requisições não 
  precisam olhar além de uma requisição, pois elas contêm todo o dado necessário 
  para entender a natureza desta, e a confibilidade pois a recuperação de falhas 
  parciais se torna mais simples \ \cite{kendall1994note}.
\item[Cache:] O servidor deve declarar quais dados podem ser guardados em \emph{caches}. 
  \emph{Caches} ajudam a reduzir a latência percebida pelo cliente, aumentam a disponibilidade 
  e confiabilidade de serviços e mitigam a carga de trabalho do servidor. Em resumo, 
  \emph{caches} reduzem todos os custos associados a um serviço na \emph{internet}.
\item[Código sob demanda:] A internet faz bastante uso de código sob demanda, que possibilita 
  que o servidor transfira executáveis, tais como \emph{scripts} e \emph{plugins}, para a 
  execução do lado do cliente. Apesar de proposto por \citeonline{fielding2000architectural}, 
  esta restrição tende a estabelecer um acoplamento de tecnologias entre servidor e cliente. 
  Por este motivo ``código sob demanda'' é a única restrição imposta pelo estilo arquitetural 
  \emph{REST} que é considerada opcional.
\end{description}

O maior desafio no projeto de uma \emph{API} é abstrair componentes do sistema
em recursos. O modelamento destes recursos estabelece os aspectos chaves da sua \emph{API}
e é semelhante ao processo de modelar um banco de dados relacional ou mesmo o modelamento 
clássico de um sistema orientado a objetos.

No processo de modelagem de recursos, rotineiramente começa-se pensando em arquétipos de 
recursos. Estes arquétipos nos ajudam a comunicar de forma consistente 
as estruturas que são frequentemente encontradas em \emph{Designs} de \emph{REST APIs}.
Idealmente, cada recurso pertence a apenas um dos seguintes arquétipos: \emph{document}, 
\emph{collection}, \emph{controller} e \emph{store}.

\emph{Document} é o arquétipo base de todos os outros, todos os outros arquétipos são especializações 
deste. Sua representação tipicamente inclui campos com dados ou \emph{hiperlinks} para outros 
recursos. \emph{Collection} é um diretório de outros recursos em que clientes podem recuperar 
ou, possivelmente, adicionar novos itens à coleção. Já \emph{Stores} são recursos que permitem 
que usuários guardem novos recursos, de forma que o próprio cliente escolha o 
identificador daquele recurso. Para recursos que se parecem mais como procedimentos que 
não se encaixam nos métodos padrões \emph{CRUD} (\emph{create, retrieve, update, delete}),
  o arquétipo \emph{Controller} é adequado. Todos estes arquétipos 
  contêm recomendações de como o identificador único deve ser escolhido, para que seja 
  claro a quem consome a \emph{API} de qual arquétipo se trata aquele recurso.

  \section{\emph{Python} e \emph{Flask}}
  Python é uma linguagem com sintaxe clara e concisa, favorecendo a legibilidade do 
  código-fonte, o que a torna uma linguagem extremamente produtiva. A linguagem incluí 
  diversas estruturas e módulos de alto nível e, por ser,
  segundo \hyperref[link:so_survey]{\emph{a pesquisa anual de desenvolvedores 
      do site \emph{StackOverflow}}},
  uma das linguagens mais populares e amadas do mundo, possuí uma infinidade 
  de excelentes \emph{Frameworks} e bibliotecas de terceiros, que podem ser instalados 
  de forma simples e gratuita. 

  A linguagem incluí vários recursos de outras linguagens 
  modernas, tais como geradores, instropecção, persistência, metaclasses e unidades de 
  teste. Multiparadigma,
  a linguagem é feita para suportar programação modular e funcional, 
  tal como orientação a objetos.
  A linguagem é interpretada através de 
  \emph{bytecode}s pela máquina virtual do \emph{Python}, tornando seu código 
  excepcionalmente portável, e, como conta com tipagem dinâmica, é muito adequada 
  para prototipação \ \cite{borges2014python}. 

  Sua sintaxe clara torna fácil seu aprendizado, mesmo por iniciantes. 
  Além de ser largamente utilizada no desenvolvimento de sistemas, a linguagem é muito utilizada 
  para automatização de tarefas, por meio de \emph{scripts}. Também se integra 
  muito bem com outras linguagens, como \emph{C} ou \emph{Fortran}, fazendo-se 
  uma ótima linguagem para a escrita de \emph{wrappers}. \emph{Python} é um 
  \emph{software} de código aberto e é largamente utilizado na indústria, 
  por grandes empresas como: \emph{Microsoft}, \emph{Yahoo}, \emph{Google}, \emph{Disney}, 
  entre tantas outras.

  \emph{Python} é largamente utilizado para a construção de aplicações \emph{Web}, normalmente 
  desenvolvidos em algum tipo de \emph{framework}, sendo \emph{Django} a escolha mais comum.
  \emph{Django} conta com todas as ferramentas necessárias para o desenvolvimento 
  embutidas em seu \emph{framework}, como \emph{ORM}, \emph{Object Relation Mapper}, 
  soluções para autenticação e autorização, e um sistema de administração. Ao oferecer
  uma solução completa, perde-se em desempenho na aplicação e inflexibiliza-se seu 
  desenvolvimento. Para aqueles 
  que querem construir uma aplicação enxuta e de forma menos opinada, sugere-se um
  \emph{microframework Web}. 

  O mais popular \emph{microframework web} escrito em Python se chama \emph{Flask}.
  Denomina-se ``\emph{micro}'' devido as poucas escolhas que foram feitas 
  em sua implementação,
  deixando a cargo do desenvolvedor a implementação dos módulos que se mostrem 
  necessários. A abordagem minimalista possibilita o surgimento de um ecossitema de 
  pacotes diverso, em que o desenvolvedor opta por pacotes que são mais adequados a 
  necessidade do projeto, ao invés de seguir decisões tomadas pelo \emph{framework}.
\section{Rust}
\emph{Rust} é uma linguagem de programação que tem como enfoque velocidade, acesso de memória 
seguro e paralelismo, sem que haja o sacrifício na expressividade de sua sintaxe. A linguagem 
é mantida e ativamente desenvolvida pela organização \emph{Mozilla}, criadora do famoso navegador 
\emph{Firefox}. 

A linguagem é compilada, fortemente tipada  e multiplataforma. Seus binários são autocontidos e não  precisam de um ambiente de \emph{runtime}. Sua velocidade é comparável ao de um programa desenvolvido em \emph{C++}.  \emph{Rust} também conta com um excelente 
gerenciador de dependências, chamado \emph{Cargo}. 

Parte do poder de \emph{Rust} vem de abstrações transpostas de linguagens puramente funcionais, como 
\emph{Haskell}. Recursos como \emph{macros}, \emph{pattern matching}, \emph{generics} e composição a partir 
de \emph{traits} são parte do porquê a linguagem é considerada tão moderna e tão amada entre seus usuários.

Outro aspecto chave da linguagem é o sistema de \emph{borrow checking}. O sistema delega ao compilador a gerência de memória sem a necessidade de um \emph{garbage collector}, o que confere maior desempenho em
tempo de execução ao código gerado. Isto é alcançado seguindo várias regras impostas pelo compilador.
Não apenas o gerenciamento de memória se beneficia desse mecanismo,
como também a programação concorrente.
Some-se a isso o incentivo que a linguagem dá à imutabilidade de dados,
e temos uma das melhores tecnologias para programação de alto desempenho \cite{klabnik2018rust}.  

\emph{Rust} é uma linguagem de sistema. Ela foi pensada para concorrer
com linguagens como \emph{C++} e \emph{C}, mas evitando muitas dificuldades
inerentes a esse nicho.

O gerenciador de pacotes e de \emph{build} \emph{Cargo}, juntamente com o
repositório de pacotes \emph{Crates} são destaque. 
A linguagem tem sido projetada pensando em ergonomia (do ponto de vista
do programador) em diversos aspectos, inclusive ferramentas.
A comunidade que a \emph{Mozilla} fomentou em torno da linguagem tem servido
de modelo no desenvolvimento de software livre.
Adotando processos inclusivos e transparentes, a adesão à comunidade
tem sido considerável.

Muitas aplicações conhecidas são feitas em \emph{Rust}, como o
\texttt{ripgrep} (o mais rápido \emph{engine} para expressões regulares),
o próprio compilador \emph{Rust}, partes do \emph{Firefox}, e muito mais.

  \section{O padrão \emph{OAuth}}
  Apesar de vivermos em um mundo que depende, cada vez mais, de serviços digitais,
  a maior parcela dos usuários ainda não se educaram ou 
  não seguem boas práticas de segurança na \emph{internet}. Não é infrequente que se 
  usem as mesmas credenciais em múltiplos serviços, mesmo que o usuário 
  não tenha meios de averiguar que suas informações são manuseadas e armazenadas de forma 
  adequada.

  Uma das possíveis soluções para uma melhor gestão de identidade na \emph{internet} 
  é o uso de identidades federadas \cite{208723}. Uma identidade federada é um meio 
  de uma pessoa vincular sua identidade eletrônica e atributos pessoais 
  a múltiplas partes distintas. Esta solução traz benefícios como mais garantias
  de segurança no armazenamento de credencias, menos sobrecarga de responsabilidade 
  para desenvolvedores de aplicações e uma melhor experiência para o usuário, já que se 
  diminui o atrito na criação e gerenciamento de múltiplas contas.

  O modelo de serviço federado pode ser dividido em duas componentes necessárias: 
  o provedor de identidade e o provedor de serviço. Ambas as partes cumprem papéis 
  distintos e sem qualquer uma delas não temos uma solução de identidade federada. 
  Para que a solução funcione, deve haver um relacionamento de confiança entre 
  o provedor de identidade e o provedor de serviço. Caso o provedor de identidade 
  não confie no provedor de serviço, este não o confiará os dados de usuários.
  Caso o provedor de serviço não confie no provedor de identidade, 
  também não confiará nas informações advindas dele.

  O provedor de identidade é quem mantém as informações de identidade. Usuários 
  se autenticam diante ao banco de credenciais do provedor, que então libera as
  informações relativas àquele usuário ao provedor de serviço que iniciou o processo 
  de autenticação. O provedor de serviço é aquele que provê serviços a usuários, 
  estes sendo aplicações, serviços de infraestrutura ou serviços de dados. Com a popularização 
  do modelo de computação em nuvem, cada vez mais usuários entram em contato com algum dos três 
  principais modelos de serviço em nuvem,  \emph{Infrastructure as a Service} (\emph{IaaS}), 
  \emph{Platform as a Service} \emph{PaaS} e \emph{SaaS}, \emph{Software as a Service}, 
  todas as quais dependem de um modelo de serviço federado \cite{rountree2012federated}.

  Existem vários padrões que implementam o modelo de serviço federado, as mais famosas sendo:
  \emph{SAML} (Security Assertion Markup Language), \emph{OAuth}, \emph{OpenID} 
  e \emph{Windows Identity Foundation}. O 
  padrão que tem ganhado mais adeptos, pela sua abordagem simples porém robusta,
  é o padrão \emph{OAuth}, que já conta com quatro revisões, sendo a última
  \emph{OAuth 2.0}. 

  \emph{OAuth 2.0} é um padrão aberto criado para suportar autorização de forma segura. 
    Existem quatro papéis definidos pelo padrão: servidor de autorização, 
  servidor de recurso, dono do recurso, comumente chamado de usuário, e o cliente, 
  que no contexto do \emph{OAuth 2.0} é uma aplicação que quer ser autorizada a 
  acessar um recurso do usuário. 

  O padrão suporta vários fluxos de autorização dependendo das capacidades 
  e grau de confiança que o servidor de autorização possui em um cliente.
  Caso o cliente tenha a capacidade de armazenar um segredo em comum com o 
  servidor de autorização, o fluxo de concessão do código de autorização \emph{(Authorization Code Grant)} 
  pode ser utilizado. Se ele não possui essa capacidade, como no caso de aplicações 
  que são executadas no computador do usuário, o fluxo implicito \emph{(Implicit Grant)} 
  pode ser utilizado. Se o cliente pode ser confiado com as credenciais do usuário, 
  comum em aplicações da mesma empresa, 
  ele pode utilizar o fluxo de senha do dono do recurso \emph{(Resource Owner Password Grant)}.

  Alem disso, o \emph{OAuth 2.0} permite que clientes obtenham permissão para acessar 
  recursos que não são relacionados a nenhum usuário utilizando o fluxo de credenciais 
  de cliente \emph{(Client Credentials Grant)}. 
  A maioria das aplicações Web, por possuirem servidores capazes de manter 
  um segredo em comum com o servidor de autorização, fazem o uso do fluxo de código de acesso,
  que é ilustrado pela \fref{fig:web_flow} e definido por \citeonline{hardt2012rfc}. 

  O fluxo de concessão de código de acesso começa quando um usuário, dono de um recurso, escolhe 
  fazer \emph{login} ou atrelar sua conta do servidor de autorização em uma aplicação. A aplicação 
  encaminha o usuário ao servidor de autorização e o usuário preenche suas credencias  
  e autoriza a aplicação a acessar um determinado 
  escopo de recursos. Em seguida, o servidor de autorização emite um código de autorização, 
  que geralmente expira em minutos. A aplicação, em posse do código de autorização, requisita 
  diretamente ao servidor de autorização um código de acesso. Com o código de acesso, a aplicação agora 
  pode acessar recursos protegidos do servidor de recursos, até que este código expire.

  Existem alguns detalhes que garantem a segurança do fluxo anterior. Quando a aplicação requisita 
  o servidor de autorização por um código de autorização, ela informa um identificador de cliente, que 
  indica sua identididade, uma \emph{url} de retorno, que é para onde o servidor de autorização mandará 
  o código de acesso e uma cadeia de caracteres aleatórios que será usado para validar se a requisição 
  veio do servidor de autorização. Caso na requisição a \emph{url} de retorno informada seja diferente 
  da cadastrada pela aplicação no servidor de autorização previamente, a requisição é abortada. No 
  recebimento da resposta do servidor de autenticação, caso a cadeia de caracteres aleátorios não seja a mesma 
  da enviada, aborta-se o processo. Para obtenção do código de acesso, a aplicação também não somente deve informar 
  ao servidor de autenticação o código de autorização, mas também um segredo, \emph{client secret},
  previamente cadastrado em conjunto 
  com seu identificador.
  \begin{figure}[htpb]
    \centering
    \includegraphics[width=0.95\linewidth]{images/web_flow.png}
    \caption{Diagrama que explica o funcionamento do fluxo de concessão do código de autorização, 
      definido na seção 4.1 da norma \emph{RFC 6749}. Fonte: \hyperref[link:oauth]{\emph{API Gateway OAuth 2.0 Authentication Flows}}.}%
    \label{fig:web_flow}
  \end{figure}
  
