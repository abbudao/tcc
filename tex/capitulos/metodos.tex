Este capítulo tem como propósito fornecer o embasamento teórico necessário 
para o entendimento da construção do sistema proposta. Aqui é contido uma
descrição detalhada das técnicas, métodos e tecnologias utilizadas, assim  
preparando o leitor para o conteúdo dos próximos capítulos.


\section{Design de APIs}
Separação de conceitos, traduzido do termo \emph{Separation of Concerns}, é uma temática 
chave na arquitetura cliente-servidor da internet. Parte do porquê a internet funciona tão bem
foi a preocupação, desde sua concepção, de uma interface uniforme que, desde que respeitada, 
 daria liberdade aos desenvolvedores de implementar, em qualquer linguagem ou tecnologia,
 um de seus componentes. 

Apesar do sucesso na separação de conceitos entre as responsabilidades que o navegador 
e o servidor empregam, a mesma preocupação não foi replicada entre interfaces e 
funcionalidades que o servidor expõe. 
Com frequência existe um alto acoplamento entre a interface, \emph{Frontend},
e o servidor, \emph{Backend}, o que resulta em uma interface não uniforme para interação 
dos recursos que deveriam estar expostos. Define-se aqui um recurso como qualquer conceito 
na internet que pode ser referenciado por um identificador único e manipulado por uma interface uniforme 
\ \cite{masse2011rest}.

A solução, que leva ao desacoplamento, é a interação com estas
funcionalidades través de uma \emph{API}, \emph{Application Programming Interface},
exposta pelo \emph{backend}, que provê formas padronizadas 
de acesso. Aqui não somente ganhamos portabilidade, já que possibilitamos que não só 
um tipo de cliente, navegadores, saibam como acessar nossos recursos, mas ganhamos espaço 
para criar uma camada de abstração do serviço \emph{Web}, modelando-o em recursos.
Estes recursos não serão projetados como uma cópia da organização de dados ou funcionalidades 
presentes no \emph{backend} mas em representações que sejam de fácil consumo e de entendimento 
ubíquo pelo lado do cliente.

O desafio de criar bons serviços na internet pode ser facilitado se empregarmos estilos 
arquiteturais já existentes. Aqui definimos estilos arquiteturais como 
um conjunto coordenado de restrições arquiteturais. \\
Um destes estilos arquiteturas que tem ganhado cada vez 
mais tração se chama \emph{REST}, \emph{Representational State Transfer}, ou em português,
\emph{Transferência Representacional de Estado}. Cunhado por \citeonline{fielding2000architectural}, 
o termo evoca como uma sistema na \emph{Web} deveria se comportar, uma máquina de estados 
virtuais em que o usuário progride através da seleção de identificadores únicos, que identificam 
recursos, e verbos \emph{HTTP}
que operam sobre estes recursos.

As restrições na arquitetura impostas pelo estilo \emph{REST} são agrupadas em seis categorias:
\begin{description}
\item[Cliente-servidor:] A separação dos papéis do cliente e servidor deve ser clara para 
  que estes possam ser projetados e implementados de forma independente. A interação 
  entre eles só acontece na forma de requisições, que são iniciadas pelo cliente. 
  Servidores devem mandar respostas apenas como reações de requisições dos clientes.
\item[Interface uniforme:] interfaces \emph{REST} possuem quatro  restrições:
identificação de recursos, manipulação de recursos através de representações, 
mensagens auto-descritivas e hipermídia como motor de estado da aplicação, \emph{HATEOAS}.\\
A primeira dita que cada recurso precisa ser enderessável 
por um identificador único, \emph{URI}, \emph{Unique Resource Identifier}.\\
Este 
recurso pode ser representado de diversas formas, seja um \emph{HTML}, que é mais adequado para 
um navegador,
ou \emph{JSON} que é mais apropriado para consumo de outro programa. Percebe-se aqui 
que a representação é apenas uma forma de interagir com o 
recurso,  não o próprio recurso. Isto é o que dita a segunda regra. \\
No consumo de um \emph{API}, o cliente especifica um recurso e seu estado 
desejado, enquanto o servidor deve responder com o recurso e seu estado real. 
Esta troca de mensagens  deve ser feita utilizando \emph{headers} e 
códigos de estado \emph{HTTP} que descrevam o estado 
do recurso e metadados correspondentes, de forma que as mensagens sejam auto-descritivas.\\
A última restrição, \emph{HATEOAS}, ajuda a aumentar a visibilidade e descoberta de recursos 
relacionados na \emph{API}, que ajudam o cliente a navegar dinamicamente nos recursos expostos.
Assim, quando referenciam-se outros recursos, também são fornecidos seus identificadores únicos 
e verbos aceitos para aquela rota.
\item[Sistema em camadas:] Restringe-se o comportamento dos componentes em camadas, em que 
  cada camada só pode interagir com subjacentes. 
  Entre as chamadas do cliente que requisita uma representação de um estado de um recurso, e 
  o servidor que processa a requisição, podem haver vários servidores entre eles. Estes 
  servidores podem prover um camada de segurança, \emph{cache}, balanceamento de carga ou 
  outras funcionalidades. Estas camadas não devem afetar a requisição ou resposta e nem o cliente 
  nem servidor precisam estar cientes se elas existem ou não.
\item[Protocolo sem estado:] O servidor não deve lembrar absolutamente nada do usuário que 
  utiliza sua \emph{API}. Isto implica que toda requisição individual precisa conter toda 
  informação necessária para que o servidor processe e retorne uma resposta. Esta restrição 
  visa aumentar a escalablidade, visibilidade e confiabilidade do sistema. Escalabilidade pois 
  permite que haja \emph{caching} de respostas e que o servidor possa desalocar recursos 
  físicos entre requisições, visibilidade pois sistemas que monitoram requisições não 
  precisam olhar além de uma requisição, pois elas contém todo o dado necessário 
  para entender a natureza desta, e a confibilidade pois a recuperação de falhas 
  parciais se torna mais simples \ \cite{kendall1994note}.
\item[Cache:] O servidor deve declarar quais dados podem ser guardados em \emph{caches}. 
  \emph{Caches} ajudam a reduzir a latência percebida pelo cliente, aumentam a disponibilidade 
  e confiabilidade de serviços e mitigam a carga de trabalho do servidor. Em resumo, 
  \emph{caches} reduzem todos os custos associados a um serviço na \emph{internet}.
\item[Código sob demanda:] A internet faz bastante uso de código sob demanda, que possibilita 
  que o servidor transfira executáveis, tais como \emph{scripts} e \emph{plugins}, para a 
  execução do lado do cliente. Apesar de proposto por \citeonline{fielding2000architectural}, 
  esta restrição tende a estabelecer um acoplamento de tecnologias entre servidor e cliente. 
  Por este motivo ``código sob demanda'' é a única restrição imposta pelo estilo arquitetural 
  \emph{REST} que é considerada opcional.
\end{description}

O maior desafio no projeto de uma \emph{API} é abstrair componentes do sistema
em recursos. O modelamento destes recursos estabelece os aspectos chaves da sua \emph{API}
e é semelhante ao processo de modelar um banco de dados relacional ou mesmo o modelamento 
clássico de um sistema orientado a objetos.

No processo de modelagem de recursos, rotineiramente começa-se pensando em arquétipos de 
recursos. Estes arquétipos nos ajudam a  comunicar de forma consistente 
as estruturas que são frequentemente encontradas em \emph{Designs} de \emph{REST APIs}.
Idealmente, cada recurso pertencerá a apenas um dos seguintes arquétipos: \emph{document}, 
\emph{collection}, \emph{controller} e \emph{store}.

\emph{Document} é o arquétipo base de todos os outros, todos os outros arquétipos são especializações 
destes. Sua representação tipicamente incluí campos com dados ou \emph{hiperlinks} para outros 
recursos. \emph{Collection} é um diretório de outros recursos em que clientes podem recuperar 
ou, possivelmente, adicionar novos items a coleção. Já \emph{Stores} são recursos que permitem 
que usuários guardem novos recursos, de forma que o próprio cliente escolha o 
identificador daquele recurso. Para recursos que se parecem mais como procedimentos que 
não se encaixam nos métodos padrões, \emph{CRUD} (\emph{create, retrieve, update, delete}),
  o arquétipo \emph{Controller} é adequado. Todos estes arquétipos 
  contém recomendações de como o identificador único deve ser escolhido, para que seja 
  claro a quem consome a API de qual arquétipo se trata aquele recurso.
  

\section{Integração e entrega contínua}
