Dedico este trabalho a três grandes mestres que me moldaram como pessoa em 
minha (tímida) trajetória. Enganam-se aqueles que pensam que um engenheiro se faz em
um curso de graduação; 
desde criança um engenheiro vem sendo construído. 

À professora Yonara Feitosa Caetano de Oliveira, agradeço por ser 
o ponto de partida desta construção; 
sua paciência e vontade de me ensinar pareciam ser
ilimitadas. Sua fome por conhecimento foi o melhor presente que poderia 
ter recebido ainda quando criança. 

Ao ilustre professor Nilson Massahiro Kemura, minha eterna gratidão não é apenas pela matemática, 
a qual tanto  aprendi em suas aulas. Suas inúmeras lições sobre ética e autocrítica compõem
grande parte da minha personalidade.

Professor Kimiyoshi Usami, o tempo que estive sob seus cuidados, em meu intercâmbio no Japão,
foram cruciais para minha formação. Foram seu apoio e encorajamento que me impulsionaram a 
persistir nesta jornada. Redescobri, na sua tutela, os prazeres do aprendizado.