\label{capitulo:cicd}
  Um dos grandes desafios no desenvolvimento de \emph{software} é criar um processo repetível e 
  confiável para entregas de aplicações. Frequentemente, lançamentos de \emph{software} são
  tratados por seus desenvolvedores como momentos estressantes. Associa-se este \emph{stress} a
  muitos processos manuais, infrequentes e propensos a erros, conduzidos em curtos prazos 
  de tempo.
  
  Pode-se, no entanto, se um processo rigoroso for seguido,  
  tornar esta tarefa fácil, tão fácil quanto o apertar de um botão.
  Para atingirmos este nível de maturidade no desenvolvimento de 
  \emph{software}, \citeonline{humble2010continuous} propõem que
  devemos seguir à risca dois princípios: automatização de todas etapas que sejam 
  automatizáveis e manter todos os artefatos necessários para um lançamento em um sistema de 
  controle de versões. 

  Idealmente, uma entrega de \emph{software} é composta por três atividades: 
  provisionar e gerenciar o ambiente em que a aplicação irá rodar (configuração de \emph{hardware},
  \emph{software}, infraestrutura e serviços externos), instalar a versão correta da aplicação 
  neste ambiente e configurar a aplicação, incluindo qualquer estado ou dado que ela possa requerer.

  De fato, até recentemente, vários destes passos para a entrega de \emph{software} pareciam 
  impossíveis ou ao menos complexas de aderirem os princípios citados por 
  \citeonline{humble2010continuous}. Como poderíamos, por exemplo, 
  versionar \emph{hardware}? Com o advento de virtualização eficiente e barata, até esta 
  tarefa que, \emph{a priori}, parecia impossível, virou algo corriqueiro e ubíquo. 
  Com advento e adoção da computação em nuvem, todos os passos necessários para automatização 
  de testes, compilação ou de artefatos e lançamento do \emph{software} estão 
  acessíveis a qualquer desenvolvedor.

  Define-se integração contínua como o processo no desenvolvimento de \emph{software} em que 
  membros de um time integram seu trabalho de forma frequente. Cada integração desencadeia etapas 
  como verificação de estilo de código, testes automatizados e construções de compilados ou 
  outros artefatos, com o objetivo de encontrar problemas nas integrações o mais rápido possível.
  Já entrega contínua leva a prática de integração contínua para outro patamar e automatiza 
  o lançamento de versões do \emph{software} dado que as etapas anteriores foram concluídas com 
  sucesso e o código foi revisado \cite{fowler2006continuous}.
